\chapter{Generative Pattern Database}
% TODO: investigate the effect of clustering data into regions before applying gen model
Three main challenges in the context of data dissemination in CIDS were identified. First, intrusion related data is usually of sensitive nature. Thus, the exchange mechanism must not compromise any policies and regulations related to data \textit{privacy}. At the same time, the usability of the data has to be preserved. Second, the data that is subject of the exchange may exhibit large volumes. That constitutes a challenge, since the dissemination is desired to be executed with \textit{minimal overhead} in a timely and scalable fashion. Lastly, the \textit{interoperability} of the CIDS with existing local IDS is an important aspect that influences the practical adoption into security architectures.

In summary, existing approaches for data dissemination mainly provide mechanisms for exchanging alert data or single attributes, e.g. IP addresses, as they focus on the correlation of intrusion detection incidents that originate from different sensors. The exchange of actual training data is neglected, possibly due to high data volumes. Thus, these systems lack of mechanisms for the extraction and global persistence of novel attack patterns, e.g. from zero day exploits, that can be used for the training of an intrusion detection sensor.

The approach that is presented in this chapter exchanges attack patterns by sharing generative machine learning models that have been trained on partitions of similar data points. Such a model-based dissemination enables the receiving side to sample a synthetic dataset that enhances existing local datasets. This provides two main advantages. First, no original data leaves a local network and therefore does not violate any privacy restrictions. Second, the data is compressed considerably by representing it in form of a generative model. In order to make that mechanism scalable, the monitored data is clustered using random projections. This way, similar data points are partitioned into globally common clusters, which is exploited as a data parallelism mechanism. Given that, bursty workloads can be served effectively in a cloud deployment. Furthermore, this mechanism enables a similarity-based correlation of distributed intrusion events. The integration of both a similarity based correlation of intrusion incidents and a mechanism for sharing attack knowledge makes it possible to extract novel patterns of distributed attacks and provide them globally within the CIDS, resulting in an improved attack detection.

While Section \ref{sec:system_architecture} gives a high-level overview of the proposed architecture and its main processing primitives, details on the specific algorithms and strategies of the main services are described in Sections \ref{sec:local_indexing} to \ref{sec:classifier_fitting}.

\section{High Level Overview}\label{sec:high_level_overview}

Several members exist in the CIDS, each of which manages an isolated IDS. Every IDS operates according to a specific set of rules, that is essentially based on the content of a local database. The goal of the generative pattern database is to close the knowledge gaps of local databases and thus increase the detection rate of associated IDSs. By providing a \textit{global view} on all local databases, individual IDSs can benefit from the collective knowledge of the CIDS.

Section \ref{subsec:example_integration} starts with a reference example to illustrate the idea that is described above and discusses the integration of the CIDS into existing infrastructures. Subsequently, Section \ref{subsec:clustering} and Section \ref{subsec:filtering_and_compression} show the key concepts that enable data distribution and correlation under the given requirements. Finally, Section \ref{subsec:global_view} combines the individual elements to present the strategy for the creation and usage of the global view.


% The global view is a collection of transformed information from local intrusion detection datasets. 

% Before any data is transferred from a local network of a member of the CIDS to the global view, it is clustered, filtered and compressed (\textit{Privacy} and \textit{Minimal Overhead}). Upon each update on the global view, a synchronization from the global view to each local view of all members is initiated.  In this way, the accumulated global knowledge is shared with individual members enriching their local databases, which form the backbone of their attack detection.


% Section \ref{subsec:high_level_architecture} explains how this approach can be integrated into existing IDS architectures and introduces the key features and techniques that enable a data dissemination and correlation which meets the challenges described in section \ref{subsec:challenges}. Subsequently, details on the rules and processes for creating the global view are presented in section \ref{subsec:global_view}.

\begin{figure}[t!]
    \centering
    \includegraphics[width=0.8\linewidth]{tikz/high_level_architecture.pdf}
    \caption{Integration of the approach into a generic NIDS.}
    \label{fig:high_level_architecture}
\end{figure}



\subsection{Example Integration}\label{subsec:example_integration}
An exemplary integration of the CIDS into a generic NIDS is shown in Figure \ref{fig:high_level_architecture}. The shown NIDS consists of three components, which can be found on the detection layer. A flow exporter computes statistical flow features based on the network packets of a switch. A discriminative model serves as a classifier that operates on the specific feature set that the exporter extracts. After completing the training of a classifier instance on a given dataset, it is deployed within the detection pipeline. There, the classifier receives a stream of network flows and predicts them.

The CIDS mainly integrates at the data level, where the training data for the classifier is provided by the \textit{local pattern database}. At this point, the local pattern database only contains the \textit{local view} of the intrusion detection data from that particular member. In order to provide a global view for that member, a synchronization process between its local pattern database and the \textit{global pattern database} has to be initiated. Before the data is transferred to the global pattern database, is subject to a set of clustering, filtering and compression operations. This way, data privacy is maintained and the overhead is minimized by reducing the data volume. On the receiving side, the global pattern database combines data from all members to a global view.

Upon each update of the global pattern database, the new state of the global view is synchronized to each local pattern database and subsequently enhances the classifier on the detection level by extending the local intrusion detection dataset. Furthermore, an identical clustering operation as on the data level is applied on the detection level. Each incoming flow is assigned to a certain cluster, which is referred to as \textit{region}. While the predictions from the classifier are suitable for detecting attacks that are known to the CIDS, regions are leveraged for the detection of novel data patterns and similarity-based correlations that uncover stealthy attacks, which are executed on the resources of multiple CIDS members simultanously.

\subsection{Clustering}\label{subsec:clustering}

Since the results of the clustering operation should be consistent, while its execution is distributed among all members of the CIDS, an unsupervised algorithm with few parameters for initialization has to be selected. Additionally, the algorithm should be scalable, since it is to be applied on whole databases on the data level and on streams of live data on the detection level. Given these requirements, random projection is a good choice. 


By using random projection, real data points are clustered according to their angular distance. Furthermore, the projection result is a binary string, which can be used for storing similar data points into a common bucket of a hash table by using the binary string as an index. In the context of the generative pattern database, the combination of random projections and hash tables is exploited as the main controlling primitive for data persistence and retrieval. Instead of using randomly selected projection planes, a shared seed results in the application of a common projection function among all members of the collaboration. In other words, similar data points from different datasets are indexed to a common global bucket, i.e. region. Each region is subject to the transformations individually, which is utilized as a data parallelism mechanism. Thus, this approach is natively suited for cloud deployments where bursty workloads can be served effectively. Lastly, this mechanism enables a similarity-based correlation of distributed intrusion events. As incoming data is monitored on the detection level, the clustering is applied, which results in a pattern that can be used for novelty checks or global occurrences within the CIDS.

\subsection{Filtering and Compression}\label{subsec:filtering_and_compression}

Two types of data are extracted within individual regions. First, metadata of local datasets is collected by counting label occurrences, which serve as indicator for determining if the respective region needs to be subject to the second extraction type. Second, models that are trained with generative algorithms on local attack data are the exchange medium for disseminating information within the CIDS. This provides two main advantages. For one, no original data leaves a local network and therefore does not violate any privacy restrictions. For another, the data is compressed considerably by representing it in form of a generative model. 

\subsection{Global View}\label{subsec:global_view}

\begin{figure}[b!]
    \centering
    \includegraphics[width=0.8\linewidth]{tikz/global_view.pdf}
    \caption{Building the global view by combining $M$ local views.}
    \label{fig:global_view}
\end{figure}


As shown in Figure \ref{fig:global_view}, each view is partitioned by a common projection function into an identical set of regions. Each local view contains original datapoints from its respective dataset. In this example, there exist three different classes globally, which occur differently in each local dataset. Examining a specific region, the combination of unique classes within all local views determines its complexity on a global level. If a region contains more than one class, it potentially exhibits a non-linear decision boundary, hence it is called complex. Otherwise, a region is called simple. Attack data within a complex region, seperated by its label, is used as training data for a generative algorithm. Subsequently, the resulting model is transferred to the global view. 


A synchronization process disseminates the region complexity estimations and the generative models to all local views. By sampling data from multiple generative models, a synthetic dataset is assembled, which is blended into the respective local datasets for enhancing the subsequent training of a discriminative model.

\section{System Architecture} \label{sec:system_architecture}

% Provide a general view on main components and their tasks and interfaces; how is this system supposed to work; how do the components interact with each other
% Specify important definitions on components and data formally


\begin{figure}[b!]
    \centering
    \includegraphics[width=1\linewidth]{tikz/detailed_architecture.pdf}
    \caption{High level CIDS architecture.}
    \label{fig:detailed_architecture}
    \end{figure}

    
    Logically, the proposed CIDS exhibits a hierarchical architecture (see Figure \ref{fig:high_level_architecture}). For one, the global infrastructure $G$ represents the collection of $M \in \mathbb{N}$ CIDS participants and their knowledge on an abstract level. For another, it provides specific services, that are globally available to each local infrastructure $L_m, m \in \{1, \dots, M\}$ that includes all CIDS components and processes within the IT infrastructure boundaries of a corresponding CIDS member. Each local infrastructure $L_m$ agrees to a specified feature extraction process that provides the monitoring data $\bm{X} \subset \mathbb{R}^d$ for the attack detection. Single data points of the monitoring data are referred to as $\bm{x} \in \bm{X}$ with a total number of features $d = |\bm{x}| \in \mathbb{N}$. In addition, the set of targets $Y \subset \mathbb{N}$ with instances $y \in Y$ is known and registered by every local infrastructure $L_m$. Furthermore, every $L_m$ prodives an individual training dataset $D_m= \{(\bm{x}_n, y_n): 1 \leq n \leq N_m\}$ of size $N_m = |D_m| \in \mathbb{N}$. CIDS communication across local boundaries occurs exclusively in a vertical direction. Thus, the exchange of information between individual $L_m$ takes place indirectly via the global pattern database $(PDB_G)$ and the global event channel $(C_G)$. Each $L_m$ includes a local pattern database $(PDB_{L_m})$, a local event channel $(C_{L_m})$ and an event-based data processing pipeline that consists of four services.

    \begin{table}[b]
        \centering
        
\begin{tabular}{ll} 
    \toprule
    \textbf{Notation} & \textbf{Description}             \\ 
    \midrule
    $G$                     & Global Infrastructure            \\
    $L_m$                   & Local Infrastructure $m$             \\
    \midrule
    $PDB_G, PDB_{L_m}$                   & Global Pattern Database, Local Pattern Database of $L_m$             \\
    $C_G, C_{L_m}$                     & Global Event Channel, Local Event Channel of $L_m$                    \\
    \midrule
    $M \in \mathbb{N}$      & Total number of CIDS participants         \\
    $m \in \{1, \dots, M\}  $         & Local Infrastructure Identifier  \\
    \bottomrule
\end{tabular}

        \caption{Summary of the architecture notation.}
    \end{table}

\subsection{Pattern Database} \label{subsec:pattern_database}

% what are the tasks of a pdb
depending on the scope (either local or global), different tasks are considered
local pdbs store intrusion detection datasets and corresponding metadata of the respective member; 
global pdbs store global metadata (combined information of local datasets) and the generative models


Each instance of a pattern database $(PDB)$ is realized as a key-value store. For the following algorithm descriptions, a $PDB$ is treated as a hash table as defined in Section \ref{subsec:hash_table}, that is referred to by replacing its function variable with the identifier of the corresponding database, e.g. $PDB_G(k)$ being a specific slot or hash value related to a key $k$ in the global pattern database. Note that if a specific hash function is already used to construct a key (e.g. Random Projection), the hash table will internally apply a distinct hash function on the key to ensure even distribution across the slots.

\subsection{Event Channel} \label{subsec:event_channel}
Event channels provide a topic-based publish-subscribe messaging mechanism that is mainly used to distribute workloads among the service instances in the processing pipeline. Via the messaging system, service instances receive and emit events, on which upon the respective operations are triggered. Changes in a pattern database result in responses that in turn are leveraged as the respective events. In this fashion, updates are propagated throughout the processing pipeline, ensuring a timely consistency among the pattern databases.

\subsection{Processing Pipeline} \label{subsec:processing_pipeline} 
Local and global pattern databases serve exclusively as data sources and sinks for operations. The only exception is the initial import of datasets $D_m$ into the \textit{Local Indexing} service via the messaging system.
\newpage
\section{Local Indexing} \label{sec:local_indexing}

The local indexing service is responsible for the preprocessing and local storage of intrusion detection datasets. As already described in Section \ref{subsec:clustering}, the data is organized in regions. This means that individual data points are first assigned to a region using a locality-sensitive hash function. Based on the generated hash value, a key is constructed that is used to persist the data point in the respective local pattern database. According to the properties of a locality-sensitive hash function, similar data points are assigned to a common region and thus form a closed processing unit for subsequent operation steps. If a region is formed or an update is made to an existing region, e.g., due to the occurrence of new data, events are emitted to inform the subsequent service. 

\begin{algorithm}
    \caption{Preprocessing and inserting $B \subset D_m$ into $PDB_{L_m}$}
    \label{alg:indexing}
    \algsetup{indent=2em}

    \begin{algorithmic}[1]
        \REQUIRE Pairs of datapoints and labels $B \leftarrow [(\bm{x}_1, y_1), \dots, (\bm{x}_b, y_b)]$
        \ENSURE Regions $R$

        \STATE $R \leftarrow \text{new Set}$
        \FORALL{$(\bm{x}, y)$ in $B$}
            \STATE $\bm{x}' \leftarrow \text{normalize}(\bm{x})$ \COMMENT{see Equation \ref{eq:normalization}}
            \STATE $r \leftarrow h(\bm{x}')$
            \STATE $k_\alpha \leftarrow \text{concatenate}(p_x, r, y)$ \label{alg:indexing_kalpha}
            \STATE $k_\beta \leftarrow g(\bm{x}')$ \label{alg:indexing_kbeta}
            \STATE $k_\gamma \leftarrow \text{concatenate}(p_y,r)$ \label{alg:indexing_kgamma}
            \IF{$PDB_{L_m}[k_\alpha]$ is None}
                \STATE $PDB_{L_m}[k_\alpha] \leftarrow \text{new Hashtable } H$
            \ENDIF
            \IF{$PDB_{L_m}[k_\gamma]$ is None}
                \STATE $PDB_{L_m}[k_\gamma] \leftarrow \text{new Set } S$
            \ENDIF
            \IF{$PDB_{L_m}[k_\alpha][k_\beta]$ is None}
            \STATE $PDB_{L_m}[k_\alpha][k_\beta] \leftarrow (\bm{x}', y)$
            \STATE insert y into Set at $PDB_{L_m}[k_\gamma]$
            \STATE $\text{insert } r \text{ into } R$
            \ENDIF         
        \ENDFOR
        \RETURN $R$
    \end{algorithmic}

\end{algorithm}

First, an intrusion detection dataset $D_m$ is sent to one or more service processors via the $C_{L_m}$. Second, the incoming stream of pairs of data points and labels $(\bm{x}_n, y_n) \in D_m$ is ingested and buffered until a batch $B$ has been accumulated. Then, the batch is preprocessed and inserted into the $PDB_{L_m}$ as described in Algorithm~\ref{alg:indexing}. In words, the datapoint $\bm{x}$ is scaled to the range $[-1, 1]$ by applying a feature-wise min-max normalization, given by

\begin{align}\label{eq:normalization}
    \bm{x}' = \frac{\bm{x}-min(\bm{X}_B)}{max(\bm{X}_B) - min(\bm{X}_B)} \cdot (b - a) + a,
\end{align}

where $a=-1$, $b=1$ and $\bm{X}_B$ is the set of data points within the batch $B$. After that, the scaled data point $\bm{x}'$ is subject to both a locality-sensitive hashing function $h$ and a non-cryptographic hashing function $g$. In this particular architecture, $h$ is a gaussian random projection with a global seed for the initialization of the projection plane $\bm{M}$ (see Section \ref{subsec:random_projection}), such that regions $r = h(\bm{x}')$ across local infrastructures are comparable.

Since the data is organized in regions, a nested scheme is applied for the insertion of pairs of datapoints and labels as depicted in Figure~\ref{fig:indexing}. In fact, the pairs within a region are further partitioned into disjoint subsets according to the label $y$. This means that for each subset of the data of a particular label within a region, a separate hashtable is initialized and inserted into the $PDB_{L_m}$ as a second level. 

Thus, for the persistence of a pair $(\bm{x}', y)$, two keys $k_\alpha, k_\beta \in K$ are constructed (see Lines~\ref{alg:indexing_kalpha}-\ref{alg:indexing_kbeta} in Algorithm~\ref{alg:indexing}). The key $k_\alpha$ is a concatenation of the prefix constant $p_x$, the bit-string $r$ and the label $y$. This way, the data is partitioned primarily by its region and secondarily by its label as described above. The key $k_\beta$ is the result of the non-cryptographic hashing function $g(\bm{x}')$, which serves as a mechanism for deduplicating identical $\bm{x}'$. 

Additionally, a third key $k_\gamma$ is constructed by concatenating the prefix constant $p_y$ and the bit-string $r$. As it is important to retrieve all existing labels within a region efficiently in a processing step of the subsequent service, $k_\gamma$ is used for storing the set of labels within a region as auxiliary metadata.

Next, if not already present, a hash table is initialized and inserted into the slot $PDB_{L_m}[k_\alpha]$. Likewise, if the slot $PDB_{L_m}[k_\gamma]$ is empty, a new set\footnote{A set describes a data structure which is essentially an unordered collection with no duplicate elements.} is initialited and inserted. After that, it is checked if the slot $H[k_\beta]$, which in turn is placed in $PDB_{L_m}[k_\alpha]$, is empty. In the positive case, the pair $(\bm{x}', y)$ is inserted into that slot and the region $r$ is inserted into the set $R$. Otherwise, no data is inserted into the local pattern database and no region is added to $R$. After processing a batch, the set of updated regions $R$ is emitted as events into $C_{L_m}$.

Since data duplicates are filtered before the insert operation in this algorithm, events are also not emitted unnecessarily multiple times if, for example, the same dataset is sent to the service repeatedly. In other words, the inserts are idempotent, which is an important property in this architecture. Since there are operations in subsequent services that are relatively computationally intensive, emitting events is expensive. For this reason, such a streaming application might in practice implement buffers at regular intervals to collect and aggregate the events of several successive batches.



\begin{figure}[t]
    \centering
    \includegraphics[width=0.8\linewidth]{tikz/indexing.pdf}
    \caption{Nested indexing in a local pattern database.}
    \label{fig:indexing}
\end{figure}


\newpage
\section{Complexity Estimation} \label{sec:complexity_estimation}

A region is said to be complex, if it contains more than one unique label. Otherwise, a region is simple. Since the data within a region already represents a cluster, the existence of multiple classes indicates a more complex decision boundary. On that basis we differentiate how a region is processed in the subsequent services of the pipeline. Furthermore, the complexity state of a region may vary, depending on the scope it is observed. Note that since the projection matrix $\bm{M}$ is initialized with the same values in every $L_m$, all hashes that were computed by $h$ are globally comparable. This means that similar data points from different datasets, e.g. $\bm{x}_i \in D_1$ and $\bm{x}_j^* \in D_2$ may be hashed to the same region $h(\bm{x}_i) = h(\bm{x}^*_j)$. However, it is also possible that that the corresponding labels $y_i \in D_1$ and $y^*_j \in D_2$ are not equal and therefore lead to a different global view on that region's complexity state. Given that, the complexity estimation module acts as a bridge between the local and global components and answers the question, which regions are considered to be complex in a global context. 

\begin{algorithm}
   \caption{Creating a global complexity state by combining local complexity states}
   \label{alg:complexity_estimation}
   \algsetup{indent=2em}

   \begin{algorithmic}[1]
         \REQUIRE Regions $R_{\text{in}}$
         \ENSURE Regions $R_{\text{out}}$
         \STATE $R_{\text{out}} \leftarrow $ new Set
         \STATE $m \leftarrow$ getID() \COMMENT{current local infrastructure identifier}
         \FORALL{$r$ in $R$}
            \STATE $k_\gamma \leftarrow \text{concatenate}(p_y, r)$
            \STATE $Y_r \leftarrow PDB_{L_m}[k_\gamma]$
            \STATE $k_\delta \leftarrow \text{concatenate}(p_y, r, m)$
            \STATE $PDB_G[k_\delta] \leftarrow Y_r$
            \STATE $S \leftarrow$ new Set
            \FORALL{$m$ in $\{1, \dots, M\}$}
               \STATE $k_\delta \leftarrow \text{concatenate}(p_y, r, m)$
               \STATE $Y_r \leftarrow PDB_G[k_\delta]$
               \STATE insert $Y_r$ into $S$
            \ENDFOR
            \IF{$|S| > 1$} 
               \STATE $c_r \leftarrow 1$ 
            \ELSE 
               \STATE $c_r \leftarrow 0$ 
            \ENDIF
            \STATE $k_\kappa \leftarrow \text{concatenate}(p_c, r)$
            \IF{$PDB_G[k_\kappa] \neq c_r$}
               \STATE $PDB_G[k_\kappa] \leftarrow c_r$
               \STATE insert $r$ into $R_{\text{out}}$
            \ENDIF
         \ENDFOR
         \RETURN $R_{\text{out}}$
   \end{algorithmic}
\end{algorithm}

First, a set of regions $R$ is received. Then, for each region $r \in R$ the following operations are defined. The set of unique labels $Y_r$ for a particular region has been stored in Algorithm~\ref{alg:indexing} as auxiliary metadata, which is now retrieved by constructing the correspoding key $k_\gamma$. Subsequently, the slot $PDB_{L_m}[k_\gamma]$ is accessed and $Y_r$ is retrieved. In the next step, $Y_r$ has to be stored in the $PDB_G$. Therefore, another key $k_\delta$ is constructed by concatenating the prefix constant $p_y$, the region $r$ and the current local infrastructure identifier $m$, which prevents the collision of information from different infrastructures.

After $Y_r$ is stored on a global level at $PDB_G[k_\delta]$, the label set information for that region from all members in the CIDS is aggregated. That aggregated view is essentially the global complexity state for that region. By iterating over all member identifiers in the CIDS, multiple keys $k_\delta$ are constructed. Each key retrieves the specific label set $Y_r$ of a member and inserts its content into the set $S$. 

After collecting all label sets in $S$, the complexity state is obtained by simply evaluating the cardinality $|S|$. If there is more than one class in a region on a global scope, that is $|S| > 1$, then assign a true value to the global complexity variable $c_r$. Otherwise, assign a false value. In order to store $c_r$, the key $k_\kappa$ is created by concatenating the prefix constant $p_c$ and the region $r$. Note, that $PDB_G[k_\kappa]$ is only updated, if storing $c_r$ changes the state that is already persisted. This is because, if an update is executed, this information has to be propagated to the next service. Thus, in that case, the region $r$ is inserted into $R_{\text{out}}$, which is subsequently sent into the global event channel $C_G$ in order to inform services in all local infrastructures about the update.
\newpage
\section{Generative Fitting} \label{sec:generative_fitting}

The generative fitting service is the most demanding procedure in the context of processing resources. The service represents the filtering and compression operations presented in Section \ref{sec:high_level_overview}. There are two scenarios based on the region's complexity. If the region is not complex, no further actions are taken except it formally was complex. Then, existing models have to be deleted, since they are not longer used. And if the region is complex, generative models are provided, which represent the exchange medium for information. More specifically, multiple \textit{Gaussian Mixture Models} (GMMs) are fitted on each label-subset of a region's data, which was stored in the indexing step in Section~\ref{sec:local_indexing}. According to a model selection process that evaluates the efficacy of each GMM, the best model is stored in the global pattern database, accessible to every member in the CIDS. The purpose of that elaborate process is that synthetic data can be sampled from these models. This way, every member has access to the global knowledge from all local infrastructures in order to enhance the local dataset that is used for fitting a classifier. Thus, this service is the key for providing \textit{privacy} and \textit{minimal overhead} while exchanging information.
% WHY GMMs? and not NNs for example?
\begin{algorithm}
    \caption{Retrieve Dataset from Region (Main Procedure)}
    \label{alg:generative_fitting}
    \algsetup{indent=2em}
 
    \begin{algorithmic}[1]
        \REQUIRE Regions $R_{\text{in}}$
        \ENSURE Regions $R_{\text{out}}$

        \STATE $m \leftarrow$ getID()
        \FORALL{$r$ in $R_{\text{in}}$}

            \STATE $k_\kappa \leftarrow \text{concatenate}(p_c, r)$
            \STATE $k_\delta \leftarrow \text{concatenate}(p_y, r, m)$

            \STATE $c_r \leftarrow PDB_G[k_\kappa]$
            \STATE $Y_r \leftarrow PDB_G[k_\delta]$ % get label set of region from global PDB

            \IF{$c_r = 0$}
                \FORALL{$y$ in $Y_r$}
                    \STATE $k_\omega \leftarrow \text{concatenate}(p_d, r, y, m)$ 
                    \STATE delete model in $PDB_G[k_\omega]$
                \ENDFOR
            \ELSE
                \STATE $L \leftarrow$ new List
                \FORALL{$y$ in $Y_r$}
                    \STATE $k_\alpha \leftarrow \text{concatenate}(p_x, r, y)$
                    \STATE $H \leftarrow PDB_{L_m}[k_\alpha]$
                    \STATE append $H$ to $L$
                \ENDFOR

                \STATE $D \leftarrow \text{preprocessing}(L)$

                \FORALL{$y$ in $Y_r$}
                    \STATE $\text{GMM} \leftarrow \text{modelSelection}(D, y)$
                    \STATE $k_\omega \leftarrow \text{concatenate}(p_d, r, y, m)$ 
                    \STATE $PDB_G[k_\omega] \leftarrow \text{GMM}$
                \ENDFOR
            \ENDIF
            


        \ENDFOR
    \end{algorithmic}
 \end{algorithm}

 First, regions that have been updated are received as events. As the data is further organized per label within a region, the labels for a region are retrieved. Then, if the region is not complex, no model fitting is executed. Instead, potentially existing models are deleted from storage. This is the case, if the complexity status of the has been changed from complex to simple.

 But if the region is complex, the generative model fitting procedure is triggered. Even if there are already models for the corresponding combination of region and label, an update of these models is initialized. For that, every hash table within a region, each containing data with a common label, is collected and added to a list. Subsequently, preprocessing operations prepare the collected region data for the model fitting. Details on the data preparation are described in Algorithm \ref{alg:data_preprocessing}.

 After that, the Model Selection process is started sequentially for each available label in the region. That way, the models are only fitted on data with the label in focus but evaluated with the complete region data. Specifics on the model selection are elaborated in Algorithm~\ref{alg:model_selection}. Finally, the best fitted model is stored in the global pattern database. So far, the main procedure has been outlined. Next, the details on the data preprocessing and the model selection are elaborated.

 \begin{algorithm}
    \caption{Preprocess Data}
    \label{alg:data_preprocessing}
    \algsetup{indent=2em}
 
    \begin{algorithmic}[1]
        \REQUIRE List of HashMaps $L$
        \ENSURE Dataset $D$
        
        \STATE $D \leftarrow \text{getValues(L)}$
        \FORALL{label in $D$}
            \IF{label $\neq 0$}
                \STATE \COMMENT{split into binary}
                \STATE $(X_a, y_a) \leftarrow (X, y)$ where $y=\text{label}$ 
                \STATE $(X_b, y_b) \leftarrow (X, y)$ where $y \neq \text{label}$
                \IF{$X.\text{shape}[0] < X.\text{shape}[1]$}
                    \STATE $X_a, y_a \leftarrow \text{upsample}(X_a, y_a)$
                \ENDIF
            \ENDIF
        \ENDFOR
        \RETURN $(X_a, y_a), (X_b, y_b)$
    \end{algorithmic}
 \end{algorithm}

 Starting with the preprocessing.

 Splitting the dataset into a binary problem, such that the label that is currently in focus is the normal class and all other classes are attack classes. That way, the generative model can be evaluated using the machine learning efficacy method in the later course. 

 The upsampling process ensures that the fitting algorithm for the GMM is able to work. In practice, the number of samples has to be at least equal to the number of components. As some label subsets of a region may exhibit a low number of samples, in extreme cases only a single sample, an upsampling process is implemented. In particular, a set of nearest neighbours is generated per sample. First, the number of samples to generate is determined by the difference of the dimensionality of the data and the number of data points. Then, in order to generate from each data point equivalently, the number of nearest neighbours to sample is determined per data points. That is, the complete number of points to resample divided by the number of samples with an interger division (resulting in an integer). In case, the result of the interger division is zero, one is added to the result. Then for each data point $x$ in the set $X$, nearest neighbours are generated by sampling from a uniform distribution. This is done by considering each feature value of $x$ individually, such that the nearest neighbour is the concatenation of the sampled feature values as $x^* = [p(x_1), p(x_2), \dots, p(x_M)]$, where $p(x_m)=\frac{1}{b-a}$ within the interval $[x_m-\delta, x_m+\delta)$. The value $\delta$ controls the interval of the uniform distribution. The larger the value for $\delta$, the further the newly generated values deviate from the orginal feature values. In Figure \ref{subfig:hist_nn} the value for a single feature $x_m$ is drawn uniformly at random. The original value of the feature was $x_m=4$, which was extended by $\delta=1\cdot10^{-2}$. By choosing a relatively small value for $\delta$, it is ensured that the generated nearest neighbours do not alter the original data distribution significantly, while enabling the subsequent fitting of the GMM for that set of data points.


 \begin{figure}
    \centering
    \begin{subfigure}[b]{0.45\textwidth}
        \centering
        \includegraphics[width=\textwidth]{tikz/uniform_distribution.pdf}
        \caption{The probability density function of a uniform distribution is $p(x) = \frac{1}{b-a}$ within the interval $[a, b)$, and zero elsewhere.}
        \label{subfig:uniform_dist}
    \end{subfigure}
    \hfill
    \begin{subfigure}[b]{0.45\textwidth}
        \centering
        \includegraphics[width=\textwidth]{tikz/histogram_nearest_neighbour.pdf}
        \caption{Histogram of \numprint{1000} samples drawn uniformly over the interval $[x-\delta, x+\delta)$ where $x=0.4$ and $\delta=1 \cdot 10^{-2}$.}
        \label{subfig:hist_nn}
    \end{subfigure}
    \caption{Nearest neighbours are generated by sampling each feature value from a the uniform distribution.}
    \label{fig:uniform_dist_nn}
\end{figure}

 \begin{algorithm}
    \caption{Upsampling}
    \label{alg:upsampling}
    \algsetup{indent=2em}
 
    \begin{algorithmic}[1]
        \REQUIRE Collection of datapoints $X$
        \ENSURE Upsampled collection of datapoints $X_{up}$
        \STATE nResample $\leftarrow X.\text{shape}[1] - X.\text{shape}[0]$ \COMMENT{Difference of dim($X$) and num($X$) to fill}
        \STATE nResamplePerX $\leftarrow$ nResample $// X.\text{shape}[0] + 1$
        \STATE $X^* \leftarrow \emptyset$
        \FORALL{$x$ in $X$}
            \STATE $x^* \leftarrow$ new Array
            \FORALL{$x_m$ in $x$}
                \STATE sample a new $x_m^*$ from $p(x_m)$ and insert into $x^*$
            \ENDFOR
            \STATE add $x^*$ to $X^*$
        \ENDFOR
        \STATE $X_{up} \leftarrow X \cup X^*$
        \RETURN $X_{up}$

    \end{algorithmic}
 \end{algorithm}

 In the next phase of the algorithm, one or more GMMS are selected for a region's data, depending on the number of unique labels within. For each label subset, multiple models are fitted within a selection process. Since the resource demands of the EM algorithm for fitting a GMM are relatively complex, the data's dimensionality is reduced by applying a principal component analysis. Later in the course, when sampling data, the inverse operation using the same PCA parameters is applied on the synthetic data and bring it back into the original dimenions. Therefore, for each label subset of a region, both the parameters of a GMM and a PCA model is stored in the global pattern database.

 Apply PCA on $X_a$, such that $99.9 \%$ of the variance of the data is preserved, then fit multiple GMM with the same data but with different parameters by the following rules; collect a list of different parameters for the number of components $C = \{2k+1: k \in \mathbb{N}, 1 \leq k \leq K\}$ with $K=\lfloor M / 2 \rfloor$, the different number of components is heuristically determined; in general, there is no exact method to determine the optimal number of components for a given dataset before fitting the model; thus, different parameters have to be tried out in a model selection method; Moreover, as stated in Section (section of GMM), the runtime of a single step of the EM algorithm in this setting is asymptotically $O(NKd^3)$ or $O(NKd^2)$ by using the incremental algorithm proposed in \cite{pinto2015fast} ($N$ data points, $K$ components and $d$ dimensions); therefore, this curse of dimensionality that is encountered in calculating the covariance matrix while fitting the GMM is coped by approximation; using the diagonal covariance matrix as approximation to the regular covariance matrix; moreover, the full covariance matrix can result in overfitting, especially on small datasets, whereas the diagonal approximation acts as a type of regularization to the model. Since the focus of this algorithm is mainly focused on providing the best model, the runtime is treated as a secondary factor; thus, both types of covariances are used within the model selection process cov$=\{\text{``full''}, \text{``diagonal''}\}$.

 After fitting a model on $X'_a = \text{PCA}(X_a)$, the first metric for the selection is calculated. Precisely, the bayesian information criterion (BIC) is calculated as in Equation(X). Note, that for the case of a diagonal covariance matrix, the number of parameters to estimate change from $\frac{d(d+1)}{2}$ to $d$, such that the BIC is given as 

 \begin{equation}
    \text{BIC}(M|D) = (Kd + d +K-1) \, \text{ln}(N) - 2 \, \text{ln}(\hat{L}).
\end{equation}



 \begin{algorithm}
    \caption{Model Selection}
    \label{alg:model_selection}
    \algsetup{indent=2em}
 
    \begin{algorithmic}[1]
        \REQUIRE Dataset $D$ with $X_a, y_a, X_b, y_b$
        \ENSURE Tuple $(\text{GMM}, \text{PCA})$ containing model parameters

        \STATE $X'_a \leftarrow \text{PCA}(X_a)$
        \STATE $C \leftarrow \{2k+1: k \in \mathbb{N}, 1 \leq k \leq K\}$ with $C=\lfloor M / 2 \rfloor$
        \STATE $B \leftarrow$ new Array
        \STATE COV $\leftarrow \{ \text{``full''}, \text{``diagonal''} \}$

        \FORALL{$k$ in $C$}
            \FORALL{cov in COV}
                \STATE GMM $\leftarrow \text{fitGMM}(X'_a)$
                \STATE GMMs$[kC] \leftarrow$ GMM
                \STATE $b_{kC} \leftarrow \text{BIC}(\text{GMM}|X'_a)$
                \STATE acc $\leftarrow$ MLefficacy$(X_a, y_a, X_b, y_b, \text{GMM}, \text{PCA})$
                \STATE $B[kC] \leftarrow (b_{kC}, \text{acc})$
            \ENDFOR
        \ENDFOR
        \STATE sort $B$ by $b_{kC}$ in $B[0]$
        \STATE sort $B$ by acc in $B[1]$
        \STATE $b_{kC} \leftarrow B[-1]$
        \RETURN $(\text{GMMs}[kC], \text{PCA}[kC])$
    \end{algorithmic}
 \end{algorithm}


 \begin{algorithm}
    \caption{Machine Learning Efficacy}
    \label{alg:ml_efficacy}
    \algsetup{indent=2em}
 
    \begin{algorithmic}[1]
        \REQUIRE $X_a, y_a, X_b, y_b, \text{GMM}, \text{PCA}$
        \ENSURE acc from DecisionTree Model $DT$

        \STATE $X^{s'}_a \leftarrow$ sample $|X_b|$ data points from $GMM$
        \STATE $X^s_a \leftarrow \text{PCA}(X^{s'}_a)^{-1}$  

        \STATE $y^s_a \leftarrow \{ 0 \}$ \COMMENT{as many 0s as the number of $X_b$}
        
        \STATE $X_{\text{train}} \leftarrow \text{concatenate}(X^s_a, X_b)$
        \STATE $y_{\text{train}} \leftarrow \text{concatenate}(y^s_a, y_b)$
    \end{algorithmic}
 \end{algorithm}
%

\newpage
\input{2_mainmatter/4_generative_pattern_database/2-4_classifier_fitting.tex}

