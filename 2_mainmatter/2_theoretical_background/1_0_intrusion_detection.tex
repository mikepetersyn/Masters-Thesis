\section{Intrusion Detection}

This section introduces the topic of intrusion detection and some important subcategories in the context of the thesis. First, IDSs are described and categorized according to different characteristics. The advantages and disadvantages of individual types of IDS are explained and examples are referred to individually. The topic of network flow monitoring is then addressed, since it is an essential component of network monitoring that is incorporated in the proposed architecture. The role of network flows in the context of network management is motivated and the typical processes of a flow exporter are described. In the context of the shortcomings of conventional IDSs for protecting large scale systems against coordinated and distributed attacks, Collaborative Intrusion Detection Systems (CIDSs) are introduced. For this purpose, it is first described what coordinated and distributed attacks are.  Then it is explained what constitutes CIDS, what requirements exist for the realization of CIDS, and what components and processes CIDS typically consist of.

\subsection{Intrusion Detection Systems}

Classic security mechanisms, such as encryption or firewalls, are considered as preventive measures for protecting IT infrastructures. However, in order to be able to react to security breaches that have already occurred, additional reactive mechanisms are required. To complement preventive measures, Intrusion Detection Systems (IDSs) have been commercially available since the late 1990s \cite[27]{whitman_principles_2018}. Generally, the main reason for operating an IDS is to monitor and analyze computer networks or systems in order to identify anomalies, intrusions or privacy violations \cite{hindy2020taxonomy}. Specifically, the following three advantages are significant \cite[391]{whitman_principles_2018}.

\begin{itemize}
    \item IDSs can detect the preliminaries of attacks, in particular the organized gathering of information about networks and defense mechanisms (attack reconnaissance), and thus enable the prevention or mitigation of damage to information assets.
    \item IDSs can help protect information assets when known vulnerabilities cannot be fixed fast enough, notably in the context of an rapidly changing threat environment.
    \item The occurrence of unknown security vulnerabilities (zero day vulnerabilities) is not predictable, meaning that no specific preparations can be made for them. However, IDSs can identify processes in the IT system that deviate from the normal state and thus contribute to the detection of zero day attacks
\end{itemize}

For an effective IDS, it is important to be able to detect as many steps as possible within the prototypical attack sequence, also called kill chain \cite[393]{whitman_principles_2018}. Since a successful intrusion into a system can be stopped at several points in this sequence, the effectiveness of the IDS increases with its functionality. The following categorization (see Figure \ref{fig:intrusion-taxonomy}) of intrusion attempts according to \cite{kendall1999database} reflects parts of that chain.

\begin{figure}[t]
    \centering
    \begin{tikzpicture}
    [
        root/.style = {draw, black},
        level 1/.style = {draw, black, sibling distance = 4cm},
        edge from parent/.style={->,black,draw}, 
        edge from parent path={(\tikzparentnode.south) -- (\tikzchildnode.north)},
        >=latex, node distance=5.2cm, edge from parent fork down
    ]%
    \node[root] {\textbf{Intrusion Type}}
        child {node[level 1, rounded corners=3pt, align=center, text width=2cm] (method) {Probing}}
        child {node[level 1, rounded corners=3pt, align=center, text width=2cm] (scope) {DoS}}
        child {node[level 1, rounded corners=3pt, align=center, text width=2cm] (scope) {R2L}}
        child {node[level 1, rounded corners=3pt, align=center, text width=2cm] (scope) {U2R}};
    \end{tikzpicture}
    \caption{A categorization of intrusions into different subtypes}
    \label{fig:intrusion-taxonomy}
\end{figure}

\textit{Probing} refers to the preambles of actual attacks, also known as attack reconnaissance. This includes obtaining information about an organization and its network behavior (footprinting) and obtaining detailed information about the used operating systems, network protocols or hardware devices (fingerprinting). 

\textit{Denial of Service} (DoS) refers to an attack aimed at disabling a particular service for legitimate users by overloading the target systems processing capacity. \textit{Remote to Local} (R2L) attacks attempt to gain local access to the target system via a network connection. And one step further, \textit{User to Root} (U2R) attacks start out with user access on the system and gain root access by exploiting vulnerabilities. Individual intrusion types can but do not have to be executed consecutively. For example, it is not mandatory to render a system incapable of action by DoS in order to subsequently gain local access to the target. However, this may be part of a strategy.

\begin{figure}[b]
    \centering
    \begin{tikzpicture}
    [
        root/.style = {draw, black},
        level 1/.style = {draw, black, sibling distance = 8cm},
        edge from parent/.style={->,black,draw}, 
        edge from parent path={(\tikzparentnode.south) -- (\tikzchildnode.north)},
        >=latex, node distance=5.2cm, edge from parent fork down
    ]%
    \node[root] {\textbf{Intrusion Detection System}}
        child {node[level 1, rounded corners=3pt, align=center, text width=3.3cm] (method) {Detection Method}}
        child {node[level 1, rounded corners=3pt, align=center, text width=3.3cm] (scope) {Detection Scope}};
    \node [below = 0.25cm of method, xshift=20pt, text width=2.8cm] (anomaly) {Anomaly-based};
    \node [below = 0.1cm of anomaly, text width=2.8cm] (misuse) {Misuse-based};
    \node [below = 0.1cm of misuse, text width=2.8cm] (hybrid) {Hybrid};
    \draw[->] (method.193) |- (anomaly.west);
    \draw[->] (method.193) |- (misuse.west);
    \draw[->] (method.193) |- (hybrid.west);
    \node [below = 0.25cm of scope, xshift=20pt, text width=2.8cm] (host) {Host-based};
    \node [below = 0.1cm of host, text width=2.8cm] (network) {Network-based};
    \node [below = 0.1cm of network, text width=2.8cm] (hybrid2) {Hybrid};
    \draw[->] (scope.195) |- (host.west);
    \draw[->] (scope.195) |- (network.west);
    \draw[->] (scope.195) |- (hybrid2.west);
    \end{tikzpicture}
    \caption{A taxonomy for IDS that distincts systems based on the utilized detection method or the  scope in which it operates}
    \label{fig:ids-taxonomy}
\end{figure}

Additionally, IDS are generally categorized by the detection scope and the employed attack detection method \cite{milenkoski2015evaluating}. Hence, a distinction is made between Host-based Intrusion Detection System (HIDS) and Network-based Intrusion Detection System (NIDS). While a HIDS resides on a system, known as host, only monitoring local activities, a NIDS resides on a network segment and monitors remote attacks that are carried out across the segment. In general, HIDS are advantageous if individual hosts are to be monitored. NIDS, on the other hand, are able to monitor traffic coming in and out of several hosts simultaneously. The disadvantage of NIDS, however, is that attacks can only be detected if they are also reflected in the network traffic. Pioneering examples in the context of HIDS are the Intrusion-Detection Expert System (IDES) \cite{lunt1992real} and the Multics Intrusion Detection and Alerting System (MIDAS) \cite{sebring1988expert}. A prominent representative in the area of NIDS is NetSTAT \cite{vigna1998netstat} \cite{vigna1999netstat}.

Furthermore, IDS are categorized into misuse-based detection and anomaly-based detection (see Figure \ref{fig:ids-taxonomy}). A misuse-based approach defines a model that describes intrusive behaviour and compares the system or network state against that model. An anomaly-based IDS, on the other hand, creates a statistical baseline profile of the system’s or network’s normal state and compares it with monitored activities. The concept of the NIDS origins from \cite{denning1987intrusion}. 

Obviously, these two opposing approaches offer different advantages and disadvantages and are capable of complementing each other. Misuse-based systems can detect malicious behavior with a low false positive rate, but assume that the exact patterns  are known. Typically, this method cannot detect novel \cite[403]{whitman_principles_2018}, metamorphic or polymorphic attacks \cite[236]{szor2005art}. An anomaly-based approach allows both known and unknown attacks to be detected, but the frequent occurrence of false-positive estimations is a major challenge. Given the amount of data that occurs in computer systems and networks nowadays, the generation of false alarms for even a fraction of this amount can render the IDS operationally unusable, since no network administrator can investigate such a large number of incidents in detail \cite{axelsson2000base}. Most implementations of misuse-based systems rely on the creation of signatures. A signature is similar to a collection of rules that describes an attack pattern. The most prominent signature-based IDS is Snort \cite{roesch1999snort}. The payload-based anomaly detector PAYL \cite{wang2004anomalous} is an example for an anomaly-based system. Hybrid approaches are conceivable in order to circumvent the respective disadvantages of different subcategories of IDS. The interested reader is referred to the following works \cite{DEPREN2005713} \cite{zhang2006hybrid} \cite{beer_hybrid2021}.

\subsection{Network Flow Monitoring}

Network monitoring is a key component for network management as network administrator derive decisions based on data that results from monitoring activities. By collecting data from network devices, such as switches, routers or clients, the current behaviour of the network is measured. Commonly, this behaviour has to meet some minimum application requirements, which can be summarized by the term Quality of Service (QoS) \cite[406]{tanenbaum2021computer}. In this context, the most early and vague definition of a network flow was introduced as "a sequence of packets traveling from the source to the destination" \cite{clark1988design}. Here, no further characteristics of a flow are given and it is not clear how one flow is differentiated from another flow. Later in the 1990s, however, the demand arose for a finer-grained view of network traffic than that provided by interface-level counters, but without the disadvantage of the huge amounts of data generated by packet tracking. Evidently, the concept of network flows has filled this gap in demand and has gained a central role in network management and research today \cite{trammell2011introduction}.

Since the first attempts at standardization in the 1990s through the efforts of the Internet Enginnering Task Force (IETF), a number of proprietary standards have emerged as each network device vendor has implemented its own flow export protocol. In order to improve interoperability in the area of network flow measurement, the IETF established the IP Flow Information Export (IPFIX) working group. This working group defined the IPFIX standard (RFC 5101), which defines the term network flow as follows.

A network flow is an aggregation of all network packets that pass an observation point of a network during a certain time interval and share a set of common properties. These properties are defined as the result of applying a function to the value of
\begin{enumerate}
    \item one or more packet, transport or application header fields,
    \item one ore more characteristics of the packet itself,
    \item one or more fields derived from packet treatment.
\end{enumerate}

This definition is loose enough to cover the range from a flow containing all packets observed at a network interface to a flow that consists only of a single packet between two applications. Note, that this definition of the set of properties is also less strict than the conventional definition of the five-tuple consisting of source IP address, source port, destination IP address, destination port and protocol \cite{rfc5101}

In general, network flows can appear either as unidirectional flow, which aggregates all packets from host A to host B (or vice versa), or as bidirectional format, that aggregates all packets regardless of direction. Depending on which flow format and flow exporter is used, additional information, e.g. statistical calculations on bytes per second, can be obtained. Other common network flow protocols are NetFlow \cite{rfc3954}, OpenFlow \cite{mck_2008} and sFlow \cite{pha_2004}.

\begin{figure}[t]
    \centering
    \begin{tikzpicture}[
    arrow/.style =  {-{Latex[length=1.5mm, width=1.0mm]},align=flush center}% 
]
    \node[draw, black, align=center] (capture) {Packet \\ Observation};
    \node[draw, black, right=2cm of capture, align=center] (metering) {Metering \\ Process};
    \node[draw, black, right=2cm of metering, align=center] (exporting) {Exporting \\ Process};
    \node[draw, black, right=2cm of exporting, align=center] (collecting) {Collecting \\ Process};

    \draw[arrow] (capture) -- (metering) node[midway,above] {Packets};
    \draw[arrow] (metering) -- (exporting) node[midway, align=center] {Flow \\ Records};
    \draw[arrow] (exporting) -- (collecting) node[midway, align=center] {IPFIX \\ Messages};
\end{tikzpicture}
    \caption{test}
    \label{fig:flow-export}
\end{figure}


Systems that generate flow data and extract additional information are called flow exporters. Generally, flow exporters are part of a general flow monitoring architecture \cite{hof_2014}, that consists of four different processes (see Figure \ref{fig:flow-export}). The \textit{packet observation} is done on interfaces of packet forwarding devices by capturing network packets and executing specific preprocessing steps like timestamping and truncation. The process of \textit{flow metering} describes the aggregation of packets into flow records. There, a set of functions, including packet header capturing, timestamping, sampling and classifying is applied on values of the packet stream listed above. Furthermore, flow records are  maintained, which may include creating, deleting or udpating records, or computing statistical flow features. After that, Fthe \textit{exporting process} places flow records in a datagram of the deployed export protocol and sends them to one or more collecting processes. Finally, the \textit{collection process} is responsible for the reception, storage and preprocessing of flow records, produced by the preceding step. Typically, the collected data is analyzed. This can be done, for example, in the context of \acrshort{ids} or traffic profiling.

Using network flows for intrusion detection has several advantages over \acrshort{dpi}. For example, \acrshort{dpi} requires highly complex and expensive infrastructure for storage and analysis of the data \cite{hof_2014}. Since flow exporter setups rely on packet header fields and statistical aggregations, a significant data reduction is achieved, which is why an analysis of network flows is scalable and also applicable for high-speed networks \cite{hof_2014}. In addition, discarding the payload results in a better compliance with data retention laws and privacy policies.

\subsection{Collaborative Intrusion Detection}

% was sind coordinated attacks

% typical characteristics of coordinated attacks

% typical attacks (scans, worms, dos, botnet)

% Wieso ein Kapitel über Coordinated Attacks im Zusammenhang mit Knowledge Sharing in CIDS? 
%   - Durch das Knowledge Sharing können insb. Informationen zu neuartigen Angriffen (z.B. Zero Days) mit anderen Mitgliedern im Verbund geteilt werden
%   - Coordinated Attacks sind häufig großflächig angelegt (large scale), sodass derselbe Angriff automatisiert auf eine große Menge von Systemen abzielt. Durch ein Teilen der Angriffsinformationen erzielt man folgende Vorteile:
%       o Teilen von vollständigen Angriffsinformationen für Datenbank: lokale Klassifizierer können neuen Angriff erkennen, da das Wissen in den Trainingsprozess integriert werden kann
%       o Teilen von Signaturen

survey on coordinated attacks in the context of collaborative intrusion detection systems\cite{Zhou2010}

However, when considering the development of the current threat environment, the effectiveness of conventional intrusion detection systems is limited. Technology trends, such as the internet of things or cloud computing, are main drivers for increasingly blurring corporate boundaries in the context of interconnection of infrastructures and shared resources. This transformation increases the potential attack surface of productive computer systems for large-scale and high-velocity cyber attacks, which traditional IDSs have limited effectiveness due to their isolated nature. For example, such stand-alone IDS will not be able to create connections between security events that occur at different infrastructures simultaneously. Due to the mentioned increase of attack surface that is related to the size of current computer networks, attackers may attempt to obfuscate the characteristic overall sequence of the intrusion by spreading single attack steps.

n order to address the aforementioned security problems of large IT systems, Collaborative Intrusion Detection Systems (CIDSs) have been proposed. In general, a CIDS is a network of several intrusion detection components that collect and exchange data on system security. A CIDS is essentially specified by two different types of components, namely the detection units and the correlation units, and their communication among each other. The detections units can be considered as conventional IDS that monitor a sub-network or a host and by that, generate low-level intrusion alerts. The correlation unit is responsible for merging the low-level intrusion alerts and their further post-processing. This includes, for instance, the correlation of the alerts, the generation of reports or the distribution of the information to the participants of the network. CIDSs pursue the following two goals \cite[24]{vasilomanolakis_collaborative_2016}.

\begin{itemize}
    \item The aggregation and correlation of data originating from different IDSs creates a holistic picture of the network to be monitored and enables the detection of distributed and coordinated attacks.
    \item CIDSs can monitor large-scale networks more effectively with the realization of a loadbalancing strategy. By sharing IDS resources across different infrastructures, short-term peak loads can be served, reducing the downtime of individual IDSs.
\end{itemize}

The requirements and key components are adopted from \cite{vas_2015}, who have summarized them within an survey. It should be noted that the structuring partly overlaps and that individual or several requirements influence each other. 

\subsubsection{Requirements}

\paragraph{Accuracy} Accuracy generally describes a collection of evaluation metrics for assessing the performance of the \acrshort{cids}. Frequently used metrics are, for example, the \acrfull{dr}, i.e. the ratio of correctly detected attacks to the total number of attacks, or the \acrfull{fpr}, which sets the number of normal data classified as an attack in relation to the total number of normal data. 

\paragraph{Minimal Overhead and Scalability} This requirement describes the operational overhead and the scalability of the system directly related to this. First, the algorithms and techniques for collecting, correlating, and aggregating data must require minimal computational overhead. Second, data distribution within the \acrshort{cids} must be as efficient. The performance required to operate the system should increase linearly with the resources used, so that computer systems of any size can be protected by the \acrshort{cids}. This property can be measured in theoretical terms by considering the complexity of the algorithms used. In practical terms, the passed time from the collection of a data point to the decision-making process can be measured.

\paragraph{Resilience} Resilience describes the ability of the system to be resistant to attacks, manipulations and system failures. A distinction is made between external attacks, such as DoS, and internal attacks by infiltrated \acrshort{cids} components. Moreover, it also covers fail-safety in general, which can be achieved, for example, by avoiding SPoFs.

\paragraph{Privacy} With regard to the protection and regulation of communication in the \acrshort{cids}, a distinction is made between internal and external communication. Data that is exchanged internally in the \acrshort{cids} network between members may contain potentially sensitive information that should not be disclosed directly to other participants for reasons of data privacy. 

This includes, among other things, legal aspects when it comes to sharing log and network data. In connection with the resilience of the system, securing communication against third parties using cryptographic methods also plays a role and represents a major challenge, especially in dynamically distributed architectures.

\paragraph{Self-configuration} This requirement describes the degree of automation of a system with regard to configuration and operation. Particularly in distributed and complex architectures, a high degree of automation is desired in order to avoid operating errors and enable automatic resolution of component failures. 

\paragraph{Interoperability} The individual components of the overall system, which were deployed in different system and network environments, should be able to interact with each other in the context of the \acrshort{cids}. In addition to system-wide standards for data collection, processing and exchange, there exists a trade-off between interoperability and privacy.

\subsubsection{Design Components}\label{ch:design_components}

\paragraph{Local Monitoring} \label{par:local_monitoring} First, a distinction is made between active and passive monitoring. Active monitoring refers to the use of honeypots that reveal themselves as attack targets in the infrastructure in order to collect attack data. Passive monitoring involves intrusion detection activity at the host or network level, which in turn can be divided into signature-based or anomaly-based methods according to the type of detection technique used (see Chapter \ref{ch:intrusion_detection_systems}). 

\paragraph{Membership Management}\label{par:membership_management} Membership management refers to ensuring secure communication channels in the form of an overlay network. Generally, membership management is categorized according to the organization and structure of the system topology. In terms of organization, there are either static approaches, in which members are added or removed manually, or dynamic approaches, in which components are organized automatically via a central entity or a protocol. Furthermore, the structure of the overlay network is relevant for the type of communication in the \acrshort{cids} network. This means that connections between the monitoring units are either centralized, hierarchical or distributed. 

\paragraph{Correlation and Aggregation}\label{par:correlation_and_aggregation} Before collected and analyzed data are shared with other participants in the \acrshort{cids} network, the data is correlated and similar data points are aggregated. The main purpose of generating global and synthetic alerts is firstly to reduce the amount of alerts overall and secondly to improve data insights. Correlation mechanisms can be categorized into single-monitor and monitor-to-monitor approaches. Single-monitor methods correlate data locally without sharing data with other monitor entities. Monitor-to-monitor approaches, on the other hand, share data with other monitoring units in order to correlate local data with shared data and thus enable insights that go beyond isolated methods. Further, a classification can be made according to the correlation techniques used, grouping four different approaches.
\begin{itemize}
    \item \textit{Similarity-based} approaches correlate data based on the similarity of one or more attributes. For instance, \cite{goo_2001} suggests using the 5-tuple information of the network data to detect duplicates within \acrshort{nids}s. The computation of similarity can be done in a variety of ways. For example, \cite{goo_2001b} defines a similarity function for each attribute and computes the overall data similarity by combining the functions using an expectation of similarity. High-dimensional data is usually problematic, as the difficulty for an effective calculation of similarity increases with the number of attributes \cite{zho_2009}.
    \item \textit{Attack scenario-based} approaches detect complex attacks based on databases that provide patterns for attack scenarios (e.g. \cite{hut_2004}, \cite{jaj_2002}). Such approaches have high accuracy for already known attacks. However, the accuracy decreases as soon as the patterns in the real data deviate from those in the database, since the definition of the scenario-rules are of a static quality in these approaches.
    \item \textit{Multistage alert correlation} aims to detect unknown multistage attacks by correlating defined pre- and post-conditions of individual alerts (e.g. \cite{cup_2002}, \cite{che_2003}). The idea behind the approach is based on the assumption that an attack is executed in preparation for a next step. The system states before and after an alert are modeled as pre- and postconditions, which are correlated with each other. While these approaches are more flexible than the all-static definition of attack scenarios, they still require the prior modeling of pre- and postconditions, which are based on expert knowledge.
    \item \textit{Filter-based} approaches filter irrelevant data in order to reduce the number of alerts within the intrusion detection context. For example, \cite{goo_2002} filters alerts by a ranking based on priorities of incidents. Priorities are calculated through comparing the target's known topology with the vulnerability requirements of the incident type. The rank that each alert is assigned to provides the probability for its validity. The accuracy of the filters is based on the quality of the description of the topology to be protected and require reconfiguration when changes are made to the infrastructure.
\end{itemize}

\paragraph{Data Dissemination}\label{par:data dissemniation} Data Dissemination describes the efficient distribution of correlated and aggregated data in the \acrshort{cids} network. How the data is disseminated is strongly influenced by the topology and membership management of the system. Centralized topologies have a predefined flow of information from the monitoring units to the central analysis unit. If the system is organized hierarchically, information flows statically or dynamically organized from lower monitoring units in the hierarchy to higher units. Distributed topologies allow the use of versatile strategies, such as flooding, selective flooding or publish-subscribe methods.

\paragraph{Global Monitoring}\label{par:global_monitoring} Global monitoring mechanisms are needed for the detection of distributed attacks which isolated \acrshort{ids}s cannot detect due to the limited information base. Thus, the detection of distributed attacks relies on the insight obtained by combining data from different infrastructures. This means that global monitoring is strongly dependent on the employed correlation and aggregation techniques. The overall strategy of the \acrshort{cids} is described by global monitoring and can have both generic and specific objectives. 

% requirements of CIDS

% design components

% verschiedene Architekturen mit Bild