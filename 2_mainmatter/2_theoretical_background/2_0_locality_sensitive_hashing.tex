\section{Locality Sensitive Hashing}
Finding similar objects, formally known as the nearest-neighbour search problem, is a central problem in computer science in general. And besides the exact search, it is also an important topic in the field of intrusion detection. For example, exact searches allow to compare the signatures of known malware with low false positive rates. Considering polymorphic attacks, however, it is essential to also detect all objects that are similar to the known attack and thus detect modified forms of it. In this context, this section presents a well-studied approach called \textit{Locality Sensitive Hashing} (LSH) that solves an approximate version of the nearest-neighbour search problem by partitioning the search space with a hash function. This way, the searching problem is reduced to pairs that are most likely to be similar. Besides the application for solving the NN search problem, for which LSH was originally conceived, the concept has been proven to be effective for numerous other use cases, such as dimensionality reduction, clustering or classification. As the proposed generative pattern database integrates a locality-sensitive hash function for enabling both similarity search and data parallelism, LSH and a specific variant called \textit{Random Projection} (RP) is described in detail in this section.

Section \ref{subsec:approximate_nearest_neighbour_problem} introduces the approximate version of the nearest neighbour search problem as it is a prerequisite for the general definition of LSH. After that, Section \ref{subsec:locality-sensitive-hashes} begins by clarifying the core idea of LSH and subsequently explains how to construct a locality-sensitive hash function. Lastly, a specific family of LSH that uses the cosine distance for similarity calculations, also known as \textit{Random Projection} is presented in Section \ref{subsec:random_projection}.

\documentclass[../../../main.tex]{subfiles}
\begin{document}

\subsection{The Approximate Nearest Neigbour Problem}\label{subsec:approximate_nearest_neighbour_problem}

For general definitions of the \gls{nn} and its variants, a set of points $\mathcal{P} = \{\bm{p}_1, \dots, \bm{p}_N \}$ in a metric space $(\mathcal{X}, d)$ with $\mathcal{P} \subset \mathcal{X}$ and where $d$ is a metric on $\mathcal{X}$, i.e., a function $d: \mathcal{X} \times \mathcal{X} \rightarrow \mathbb{R}$ is considered. It is assumed that $d$ is a proper metric, which means that it is

\begin{enumerate}
    \item \textit{symmetric}: $d(\bm{p},\bm{q})=d(\bm{q},\bm{p})$,
    \item \textit{reflexive}: $d(\bm{p},\bm{q}) \leq 0$, $d(\bm{p},\bm{q})=0 \iff \bm{p}=\bm{q}$ and satisfies the 
    \item \textit{triangle inequality}: $d(\bm{p},\bm{q}) \leq d(\bm{p},\bm{s}) + d(\bm{s},\bm{q})$.
\end{enumerate}

Given these assumptions, the authors in \cite{indyk_approximate_1998} define the \gls{nn} as stated in Definition~\ref{def:nn}. As the algorithms in the presented approach operate on vectors of statistical flow features (see Section~\ref{subsec:network_flow_monitoring}), the input set is restricted to $\mathbb{R}^D$. In particular, the input data is considered to be high dimensional, such that $D \gg 1$, and could define $d$ as, e.g., the euclidean distance. In this case, an exhaustive search would require a query time of $O(D \cdot N)$. Unfortunately, all exact algorithms that provide a better time complexity than an exhaustive search require $O(2^D)$ space \cite{rubinstein2018hardness}.

\begin{definition}[\Acrlong{nn}\cite{indyk_approximate_1998}]\label{def:nn}
    Construct a data structure so as to efficiently answer the following query: Given any query point $\bm{q} \in \mathcal{X}$, find some point $\bm{p} \in \mathcal{P}$ such that

   \begin{equation}
       \mathop{\text{min}}_{\bm{p} \in \mathcal{P}} \; d(\bm{p}, \bm{q}).
   \end{equation}
\end{definition}

This tradeoff between time and space complexity is usually referred to as ``curse of dimensionality'' and can only be resolved by accepting approximate solutions. In \cite{indyk_approximate_1998} the \gls{cnn} is defined as follows.

\begin{definition}[$c$-Approximate nearest-neighbour search problem]
    For any given query point $\bm{q} \in \mathcal{X}$ and some approximation factor $c > 1$, find some point $\bm{p} \in \mathcal{P}$ such that
    \begin{equation}
        d(\bm{p},\bm{q}) < c \cdot \mathop{\text{min}}_{\bm{s} \in P} \; d(\bm{s}, \bm{q}).
    \end{equation}
\end{definition}

Thus, the distance from the query point $\bm{q}$ to the approximate nearest neighbour $\bm{p}$ is at most $c$ times the distance to the true nearest neighbour $\bm{s}$. However, \gls{lsh} does not solve the \gls{cnn} directly. Instead, the authors in \cite{indyk_approximate_1998} relax the problem by introducing the \gls{crnn}.

\begin{definition}[$cr$-Approximate nearest-neighbour search problem]
    For any given query point $\bm{q} \in \mathcal{X}$, some approximation factor $c > 1$ and some target distance $r > 0$, if there exists a point $\bm{p} \in \mathcal{P}$ where $d(\bm{p},\bm{q}) \leq r$, then return a point $\bm{p}' \in \mathcal{P}$ where
    \begin{equation}
        d(\bm{p}',\bm{q}) \leq cr.
    \end{equation}
\end{definition}

Figure~\ref{fig:nearest_neighbour} depicts the \gls{crnn}. The distance $r$ represents the distance of the query object from its nearest neighbour. If there is such a point, the algorithm returns points within $cr$ distance from the query object, else it returns nothing. In \cite{indyk_approximate_1998} it is shown that LSH can solve the \gls{cnn} by solving the \gls{crnn} for different settings of $r$ .

\begin{figure}[t!]
    \centering
    \includestandalone{2_mainmatter/2_preliminaries/2_locality_sensitive_hashing/tikz/nearest_neighbour}
    \caption{In the \acrlong{crnn} some point within $cr$ is accepted, if there exists a point $\bm{p}$ where $d(\bm{p},\bm{q}) \leq r$.}
    \label{fig:nearest_neighbour}
\end{figure}

\end{document}

\subsection{Locality-Sensitive Hash Functions} \label{subsec:locality-sensitive-hashes}

Introduced by Indyck and Motwani in 1998 \cite{indyk_approximate_1998} as an algorithm that solves the approximate nearest neighbour problem (ANN), locality-sensitive hashing (LSH) has since been extensively researched and is now considered among the state of the art for approximate searches in high-dimensional spaces.\footnote{See \cite{nagarkar2021exploring} for an exhaustive survey of NN-Search Techniques.} The basic idea of the approach is to partition the input data using a hash function that is sensitive to the location of the input within the metric space. This way, similar inputs collide with a higher probability than inputs that are far apart. Thus, LSH exhibits fundamental differences to conventional hash functions \footnote{In the following, cryptographic and non-cryptograhpic hash functions are referred to as conventional hashing.}, although the most general definition applies to both.

A hash function is a function that maps a large input set to a smaller target set. The elements of the input set are called \textit{messages} or \textit{keys} and may be of arbitrary different lengths. The elements of the target set are called \textit{digests} or \textit{hash values} and are of fixed size length.

More specifically, we define a hash function as $h: A \rightarrow B : \bm{p} \mapsto \bm{u}$ where $A \subset \mathbb{R}^d$ with $d \in \mathbb{N}$ is the input set and $B=\{0, 1\}^k$ with $k \in \mathbb{N}$ the target set of all bit sequences of fixed size $k$, with $k < d$. According to the notation in \cite[256]{cormen2022introduction}, a hash table is defined by a hash function $h$, that maps the keyspace $K$ into the slots of a hash table $T[0 \,.\,.\, S-1], S \in \mathbb{N}$, i.e. $h: K \rightarrow \{0, 1, \dots, S-1\}$ with $S \ll |K|$. Thus, a message with key $k \in K$ hashes to slot $h(k)$. Additionally, $h(k)$ is the hash value of the key $k$.

Typically, conventional hashing is used for the realization of, e.g. hash tables, data integrity checks, error correction methods or database indexes. Depending on the application, different requirements are imposed on the utilized hash function. In this context, the most important property of a hash function is the probability of a \textit{collision}. A collision occurs when two keys $\bm{p}_1 \neq \bm{p}_2$ are projected onto the same hash value $\bm{u} = h(\bm{p}_1) = h(\bm{p}_2)$. For example, cryptographic hash functions are used in systems, where adversaries try to break such systems. Thus, different security requirements are defined for cryptograhpic hash functions \cite[349]{williamcryptography}. In particular, these hash functions are designed to be resistant against collisions, which is the key difference to LSH. Applications that do not require the hash function to be resistant against adversaries, e.g. hash tables, caches or de-duplication, are usually implemented by using a hash function that exhibits relaxed guarantees on the security properties in exchange for significant performance improvements. However, locality-sensitive hashes behave differently as illustrated in Figure \ref{fig:hashing_differences}. LSH projects similar inputs onto the same or near elements. In contrast, conventional hashing tries to distribute projections onto the elements of the target as randomly as possible.

\begin{figure}[t!]
    \centering
    \includegraphics[width=0.8\linewidth]{tikz/hashing_differences.pdf}
    \caption{The behaviour of conventional hashing and LSH}
    \label{fig:hashing_differences}
\end{figure}

As a locality-sensitive hashing function can be constructed as a general concept, specific families of functions can be derived. We refer to a \textit{family} of hash functions $\mathcal{H}: A \rightarrow B$ as a collection of hash functions that have the same domain and range, share a basic structure and are only differentiated by constants. Three basic requirements are demanded for such a family of functions \cite[99]{leskovec_rajaraman_ullman_2014}.

\begin{enumerate}
    \item Close pairs of input should be hashed in to the same bucket with higher probability than distant pairs.
    \item Specific functions of a family need to be statistically independent, such that the FPR and FNR can be improved by combining two or more functions.
    \item Functions need to be efficient, i.e., be faster compared to an exhaustive search.

\end{enumerate}

The first step is to define LSH generally. Applied on the $cR$-ANN, the first requirement states, with high probability, two points $p$ and $q$ should hash to the same hash value if their distance is at most $R$, i.e., $d(p,q) \leq R$. And if their distance is at least $cR$, the points should hash to different hash values, i.e. $d(p,q) > cR$. Thus,  A formal definition of a locality-sensitive hash function is given as follows \cite{andoni2006near}.

\begin{definition}[Locality-Sesitive Hash Function]
    Given a target distance $R \in \mathbb{R}, R>0$, an approximation factor $c \in \mathbb{R}, c>1$ and probability thresholds $P_1, P_2 \in \mathbb{R}$, a family $\mathcal{H} = \{h: A \rightarrow B$\} is called $(R, cR, P_1, P_2)$-sensitive if for any two points $p,q \in A$ and any hash function $h$ chosen uniformly at random from $\mathcal{H}$ the following conditions are satisfied:
        \[
        \setlength\arraycolsep{0pt}
        \renewcommand\arraystretch{1.2}
            \begin{array}{LCL}
                d(p,q) \leq R & \implies & P[h(p)=h(q)] \geq P_1, \\
                d(p,q) \geq cR & \implies & P[h(p)=h(q)] \leq P_2.
            \end{array}
        \]
\end{definition}
 
Ideally, the gap between $P_1$ and $P_2$ should be as big as possible as depicted in Figure \ref{subfig:exact_probability}, which in fact represents an exact search. This is, as already discussed, no option due to its time and space requirements. Considering a single locality-sensitive function as shown in Figure \ref{subfig:lsh_probability}, where the probability gap between $P_1$ and $P_2$ is relatively close, the false negative rate would be relatively high. Increasing the gap would require to increase $c$ and lead to a high number of false positives. Therefore, a single function would provide only a tradeoff. But it is possible to increase $P_1$ close to $1$ and decrease $P_2$ close to $1/n$ while keeping $R$ and $cR$ fixed as shown in Figure \ref{subfig:lsh_probability_boosted} by introducing a process called \textit{amplification}. Two different forms, namely the AND-construction and the OR-construction, can be applied.

\begin{figure}
    \centering
    \begin{subfigure}[b]{0.3\textwidth}
        \centering
        \includegraphics[width=\textwidth]{tikz/lsh_probability.pdf}
        \caption{Single LSH Function.}
        \label{subfig:lsh_probability}
    \end{subfigure}
    \hfill
    \begin{subfigure}[b]{0.3\textwidth}
        \centering
        \includegraphics[width=\textwidth]{tikz/lsh_probability_boosted.pdf}
        \caption{Amplified LSH Function.}
        \label{subfig:lsh_probability_boosted}
        \end{subfigure}
        \hfill
        \begin{subfigure}[b]{0.3\textwidth}
            \centering
            \includegraphics[width=\textwidth]{tikz/exact_probability.pdf}
            \caption{Exact Search}
            \label{subfig:exact_probability}
    \end{subfigure}
    \caption{The behaviour of a $(R, cR, P_1, P_2)$-sensitive function in (\ref{sub@subfig:lsh_probability}) and (\ref{sub@subfig:lsh_probability_boosted}) (adapted from \cite[100]{leskovec_rajaraman_ullman_2014}) approaching the ideal probability gap in (\ref{sub@subfig:exact_probability}) resembling the behaviour of an exact search.}
    \label{fig:lsh_probability}
\end{figure}
   
By applying a logical AND-construction on $\mathcal{H}$ the threshold $P_2$ is reduced. For that, sample $K$ hash functions indepedently from $\mathcal{H}$ and hash each point $p \in P$ to a $K$-dimensional vector with a new constructed function $g \in \mathcal{H}^K$ as

\begin{equation}\label{eq:or_construction}
    g(p) = [h_1(p), h_2(p), \dots, h_K(p)].
\end{equation}

Since all hash functions $\{h_1, \dots, h_K\} \in \mathcal{H}$ are statistically independent the product rule applies and for any two points $p$ and $q$, a collision occurs if and only if $h_i(p)=h_i(q)$ for all $k \in \{1, \dots, K\}$. The probability gap is then defined as follows:

\begin{align}
    P[h_i(p)=h_i(q)] \geq P_1 \implies P[g(p)=g(q)] \geq P_1^K \\
    P[h_i(p)=h_i(q)] \leq P_2 \implies P[g(p)=g(q)] \leq P_2^K.
\end{align}

Thus, by increasing $K$ the threshold $P_2$ can be arbitrarily decreased and approaches zero. However, the AND-construction lowers both $P_1$ and $P_2$. In order to improve $P_1$, the OR-construction is introduced. For that, a number of $L \in \mathbb{N}$ functions $\{g_1, \dots, g_L\}$ are constructed, where each function $g_l, \; l \in \{1, \dots, L\}$ stems from a different family $\mathcal{H}_l$. Note, that the algorithm is successful, when the two points $p, q$ collide at least once for some $g_l$. Therefore, hashing a point $p \in P$ with each $g_l$ results in the following probability for collision.

\begin{align}
    P[\exists l, g_l(p)=g_l(q)] &= 1 - P[\forall l, g_l(p) \neq g_l(q)] \\
                                &= 1 - P[g_l(p) \neq g_l(q)]^L \\
                                &\geq 1 - (1-P_1^K)^L
\end{align}

As the AND-construction lowers both $P_1$ and $P_2$, similarly the OR-construction rises both thresholds. By choosing $K$ and $L$ judiciously, $P_2$ approaches zero while $P_1$ approaches one. Both constructions may be concatenated in any order to manipulate $P_1$ and $P_2$. Of course, the more constructions are used and the higher the values for the parameters $K$ and $L$ are picked, the more time is considered for the application of the function. 

The general algorithm for solving the $cR$-ANN using LSH consists of two consecutive steps. First, a set of points is required for the construction of the data structrue as described in Algorithm \ref{alg:lsh_preprocess}. For each function $g_l$ a corresponding hash table $T_l$ is initialized that will store all data points $p_n$. In other words, all data points are preprocessed and stored $L$ times.The resulting data structure represents the database, which is searched through, when answering a query. Note, that a common collision handling method, such as seperate chaining for example, is applied if a certain slot is already occupied. 

\begin{algorithm}
    \caption{LSH Preprocessing}
    \label{alg:lsh_preprocess}
    \algsetup{indent=2em}
    \begin{algorithmic}[1]
        \REQUIRE $N$ points $p_n \in P$ with $n \in \{1, \dots, N\}, N \in \mathbb{N}$
        \ENSURE data structure $\{T_1, \dots, T_L\}$
        \STATE construct hash functions $g_1, \dots, g_L$ each of length $K$ (see Equation \ref{eq:or_construction})
        \FORALL{$g_l \in \{g_1, \dots, g_L\}$}
            \STATE $T_l \leftarrow$ new HashTable
            \FORALL{$p_n \in P$}
                \STATE add $p_n$ to $T_l[g_l(p_n)]$
            \ENDFOR
        \ENDFOR
    \end{algorithmic}
 \end{algorithm}

 Answering a query point $q$ is done by hashing it multiple times for each $g_l$ as in the preprocessing step (see Algorithm \ref{alg:lsh_query}). Each time, a set of points $P$, stored in the correspoding hash table $T_l$ at the slot $g_l(q)$, are retrieved. That is, identify all $p_n$, such that $g_l(p_n) = g_l(q)$. For each identified $p_n$, a distance function evaluates if the distance $d(p_n, q)$ is within the search perimeter $cR$. If positive, the point is added to the result set $S$, which is returned at the end of the algorithm.

 \begin{algorithm}
    \caption{LSH Query}
    \label{alg:lsh_query}
    \algsetup{indent=2em}
    \begin{algorithmic}[1]
        \REQUIRE a query point $q$ and data structure $\{T_1, \dots, T_L\}$
        \ENSURE a set $S$ of nearest neighours of $q$
        \STATE $S \leftarrow \emptyset$
        \FORALL{$g_l \in \{g_1, \dots, g_L\}$}
            \STATE $P \leftarrow T_l[g_l(q)]$
            \IF{$|P| > 0$}
                \FORALL{$p_n \in P$}
                    \IF{$d(p_n,q) \leq cR$}
                        \STATE add $p_n$ to $S$
                    \ENDIF
                \ENDFOR
            \ENDIF
        \ENDFOR
        \RETURN $S$
    \end{algorithmic}
 \end{algorithm}

 With regard to preprocessing, the consideration of space complexity of Algorithm \ref{alg:lsh_preprocess} is interesting, whereas the runtime of Algorithm \ref{alg:lsh_query} is the most relevant when examining the execution of queries. Fortunately, lower bounds on both quantities have been proven in \cite{motwani2006lower}, resulting in the following theorem.

\begin{theorem}
    Let $(X,d)$ be a metric on a subset of $\mathbb{R}^d$. Given a $(R, cR, P_1, P_2)$-locality sensitive hash family $\mathcal{H}$ and write $\rho = \frac{\text{log}(1/P_1)}{\text{log}(1/P_2)}$. Then for $n = |X|$ and for any $n \geq \frac{1}{P_2}$ there exists a solution to the $cR$-ANN with space complexity $O(dn+n^{1+\rho})$ and query time of $O(n^{\rho})$.
\end{theorem}

Therefore, it follows that the query time of LSH can only be sublinear, if $\rho < 1$, which is the case if the inequality $P_1 > P_2$ is satisfied.



\documentclass[../../../main.tex]{subfiles}
\begin{document}
\subsection{Random Projection}\label{subsec:random_projection}

This section presents a specific technique that is a locality-sensitive hashing function. The algorithm that bases on this technique is known as Gaussian Random Projection (GRP). This approach utilizes the cosine similarity of two real vectors in order to determine their similarity. In the course of this section, basic definitions for Cosine Distance and Cosine Similarity are introduced. Then, the individual calculation steps of the approach are specified. Finally, a visual proof verifies that \gls{grp} is a form of \gls{lsh}.
\newpage
The cosine distance can be applied in euclidean spaces and discrete versions of euclidean spaces \cite[95]{leskovec_rajaraman_ullman_2014}. For two real vectors $\bm{p}_1$ and $\bm{p}_2$, the cosine distance between is equal to the the angle between $\bm{p}_1$ and $\bm{p}_2$, regardless of the dimensionality of the space. Note, that by applying the arc-cosine function, the result is in the range $[0, 180]$. The following definition formalizes what has been stated so far.

\begin{definition}[Cosine Distance]
    Given two vectors $\bm{p}_1$ and $\bm{p}_2$, the cosine distance $\theta(\bm{p}_1, \bm{p}_2)$ is the dot product of $\bm{p}_1$ and $\bm{p}_2$ divided by their euclidean distances from the origin ($L_2$-norm):
    \begin{equation}
        \theta(\bm{p}_1, \bm{p}_2) = \text{cos}^{-1} \bigg( \frac{\bm{p}_1^\top \bm{p}_2}{||\bm{p}_1|| \: ||\bm{p}_2||} \Bigg).
    \end{equation}
\end{definition}

The angle $\theta$ can be normalized to the range $[0, 1]$ by dividing it by $\pi$. This way, the cosine similarity is simply given by the following definition.

\begin{figure}[t!]
    \centering
    \includestandalone{2_mainmatter/2_preliminaries/2_locality_sensitive_hashing/tikz/random_projection.tex}
    \caption{Illustration of a random hyperplane partitioning the space}
    \label{fig:rp_3d}
\end{figure}

\begin{definition}[Cosine Similarity]
    The cosine similarity is computed as
    \begin{equation}
        1- \frac{\theta(\bm{p}_1, \bm{p}_2)}{\pi}
    \end{equation}
\end{definition}

Introduced in \cite{charikar2002similarity}, the \gls{grp} is defined as follows. Given a point $\bm{p} \in \mathcal{P} \subset \mathbb{R}^D$ and a randomly selected hyperplane defined as $\bm{M}=(a_{ij}) \in \mathbb{R}^{D \times K}$ where $a_{ij} \sim \mathcal{N}(0, I)$, a \textit{gaussian random projection (GRP)} aims to (\RomanNumeralCaps{1}) reduce the dimensionality from $D$ to $K$ dimensions and (\RomanNumeralCaps{2}) provide a binary encoding by first projecting $\bm{p}$ onto $\bm{M}$ and subsequently applying the sign function to each element of the result, which can be formalized as

\begin{gather}\label{eq:grp_sign}
    h(\bm{p}) = [h(\bm{p}, a_1), \dots, h(\bm{p}, a_K)] \text{ with } h(\bm{p}, a_k) = sign(\bm{p}^\top a_k) \\
    \text{with } sign(x) = \Biggl\{ \begin{array}{lc}
        0 & \text{if } x < 0, \\
        1 & \text{if } x \geq 0.
    \end{array}
\end{gather}

An illistration of a random hyperplane that dissects the three-dimensional space and partitions the data space is given in Figure \ref{fig:rp_3d}. The resulting digest is a binary vector $h(\bm{p}) = \bm{b} \in \mathcal{B}$ with $\mathcal{B} = \{0, 1\}^K$ that is used as bucket index for storing $\bm{p}$ in a hash table. For any two messages $\bm{p}_1, \bm{p}_2$, the probability of being hashed to the same bucket increases with a decreasing distance, given as
\begin{equation}\label{eq:rp_proba}
    P[h(\bm{p}_1) = h(\bm{p}_2)] = 1 - \frac{\theta(\bm{p}_1, \bm{p}_2)}{\pi}.
\end{equation}

For a visual proof of the claim in Equation \ref{eq:rp_proba} consider Figure \ref{fig:rp_2d}, where two vectors $\bm{p}_1$ and $\bm{p}_2$, regardless of their dimensionality, define a plane and an angle $\theta$ in this plane. Pick a hyperplane $E_1$ that intersects the plane that is spanned by $\bm{p}_1$ and $\bm{p}_2$ in a line (depicted as a red dashed line), by randomly selecting a corresponding normal vector $n_1$. Since $\bm{p}_1$ and $\bm{p}_2$ are on different sides of the hyperplane, their projections given by $\bm{p}_1^\top n_1$ and  $\bm{p}_2^\top n_1$ will have different signs. Such a scenario, where two points have different signs, is interpreted as a notion of dissimilarity. Thus, the hyperplane acts as a clustering primitive that partitions the original input set into two disjunct sets.

The opposite scenario is illustrated by a random normal vector $n_2$ that is normal to the hyperplane $E_2$, which is represented by the blue dashed line. Both $\bm{p}_1^\top n_2$ and  $\bm{p}_2^\top n_2$ will have the same sign. This in turn is interpreted as a notion of similarity, since both points are clustered into the same set. All angles between the intersection line of the random hyperplane and the plane spanned by $\bm{p}_1$ and $\bm{p}_2$ are equally likely. Thus, the probability that the hyperplane looks like the red line is $\theta / \pi$ and like the blue line $1 - \theta / \pi$.

Therefore, the \gls{grp} is a $(r, cr, (1-r/\pi), (1-cr/\pi))$-sensitive family for any $r$ and $cr$. As already explained in Section \ref{subsec:locality-sensitive-hashes}, such a family can be amplified as desired. The AND-construction is refers to the selection of a number of $K$ different random hyperplanes $r_k$. According to the construction in Equation \ref{eq:grp_sign}, all projections $sign(\bm{p}_n^\top n_k)$ for a single point are collected in a $K$-bit vector. Likewise, the OR-construction refers to the initialization of a number of $L$ such AND-constructions. The algorithmic procedures for preprocessing and queries from Algorithm \ref{alg:lsh_preprocess} and Algorithm \ref{alg:lsh_query} can be applied without change. 

\begin{figure}[t]
    \centering
    \includestandalone{2_mainmatter/2_preliminaries/2_locality_sensitive_hashing/tikz/random_hyperplane_2d}
    \caption{Visual Proof of claim in equation \ref{eq:rp_proba}}
    \label{fig:rp_2d}
\end{figure}
\end{document}