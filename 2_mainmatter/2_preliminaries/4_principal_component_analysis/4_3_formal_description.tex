\documentclass[../../../main.tex]{subfiles}
\begin{document}

\subsection{Formal Description}

Consider a dataset $\mathcal{X} = \{\bm{x}_1, \dots, \bm{x}_N\} \in \mathbb{R}^D$ with mean $\bm{0} = [0_1, \dots, 0_N]^\top$ and a covariance matrix

\begin{equation*}
    \bm{\Sigma} =\frac{1}{N}\sum_{n=1}^N\bm{x}_n\bm{x}_n^\top.
\end{equation*}

In terms of dimensionality reduction, a compressed representation of the dataset $\mathcal{X}$, where each datapoint is defined as

\begin{equation*}
    \bm{z_n} = \bm{B}^T\bm{x}_n \in \mathbb{R}^M
\end{equation*}

is searched for, which retains as much signal as possible \cite{Hotelling1933AnalysisOA}. This is done by finding a linear transformation

\begin{equation*}
    \bm{B}=[\bm{b}_1, \dots, \bm{b}_M] \in \mathbb{R}^{D\times M}
\end{equation*}

that projects $\mathcal{X}$ onto the subspace that is spanned by the columns of $\bm{B}$. Thus, $\bm{B}$ is an orthonormal basis. In order to preserve the most variance of the data, the basis vectors of $\bm{B}$ must point into the direction of maximal variance in the data, which is found by computing the eigenvectors of the covariance matrix $\bm{\Sigma}$ of $\mathcal{X}$. The projection of the data onto a lower-dimensional subspace is equal to the eigenvalue that is associated with the basis vector that spans this subspace. 

Therefore, maximizing the variance of the low-dimensional representation requires to choose the basis vector that is associated with the largest eigenvalue of the data covariance matrix. This eigenvector is called the first principal component. All other directions in $\mathbb{R}^M$, which both maximize the variance and are orthogonal to all the other directions are the remaining $M-1$ principal components.

As stated before, the projection matrix $\bm{B}$ is obtained by diagonalizing the covariance matrix $\bm{C}$, which can be done with via eigendecomposition. In the following, it is shown how the eigendecomposition is related to the diagonalization of a matrix and that an eigendecomposition of a covariance matrix is guaranteed to exist.

\begin{theorem}[Matrix Diagonalization] \cite[p.98]{dei_2020}
A matrix $\bm{A} \in \mathbb{R}^{N \times N}$ is diagonalizable if it is similar to a diagonal matrix, i.e., if there exists an invertible matrix $\bm{P} \in \mathbb{R}^{N \times N}$ such that $\bm{D} = \bm{P}^{-1} \bm{A} \bm{P}$.
\end{theorem}

\begin{theorem}[Eigendecomposition]\label{th:eigendecomposition} \cite[p.99]{dei_2020}
A square matix $\bm{A} \in \mathbb{R}^{N \times N}$ can be factorized into
\begin{equation*}
    \bm{A} = \bm{P}\bm{D}\bm{P}^{-1},
\end{equation*}
where $\bm{P} \in \mathbb{R}^{N \times N}$ and $\bm{D}$ is a diagonal matrix whose entries are the eigenvalues of $\bm{A}$, if and only if the eigenvectors of $\bm{A}$ form a basis of $\mathbb{R}^n$.
\end{theorem}

It follows, that only non defective matrices can be diagonalized. In other words, a square matrix that does not have a complete basis of eigenvectors is not diagonalizable. However, the following theorem confirms the existence of such a basis for symmetric matrices.

\begin{theorem}[Spectral Theorem] \cite[p.94]{dei_2020}
If $\bm{A} \in \mathbb{R}^{N \times N}$ is symmetric, there exists an orthonormal basis of the corresponding vector space consisting of eigenvectors of $\bm{A}$, and each eigenvalue is real.
\end{theorem}

From the fact that every covariance matrix is symmetric, the following can be concluded.

\begin{theorem}[Diagonalization of a Symmetric Matrix]
A symmetric matrix $\bm{S} \in \mathbb{R}^{n \times n}$ can always be diagonalized.
\end{theorem}

Therefore, it is assured that an eigendecomposition of a symmetric matrix $\bm{A}$ exists (with real values), and that an \acrshort{onb} of eigenvectors can be found, so that $\bm{A} = \bm{P}\bm{D}\bm{P}^{-1}$, where $\bm{D}$ is diagonal and the columns of $\bm{P}$ contain the eigenvectors. Rewriting the equation from Theorem \ref{th:eigendecomposition} as follows reveals the eigenvalue equations:

\begin{equation*}
        \bm{A} [p_1, \dots, p_N] = [p_1, \dots, p_N] \left[\begin{array}{ccccc}
            \lambda_1 & 0 & 0 & \cdots & 0 \\
            0 & \lambda_2 & 0 & \cdots & 0 \\
            0 & 0 & \lambda_3 & \cdots & 0 \\
            \vdots & \vdots & \vdots & \ddots & \vdots \\
            0 & 0 & 0 & \cdots & \lambda_N
        \end{array}\right] \Longleftrightarrow \begin{array}{ccc} 
            \bm{A}p_1 &= &\lambda_1p_1\\
            \bm{A}p_2 &= &\lambda_2p_2\\
            \bm{A}p_3 &= &\lambda_3p_3\\
            & \cdots& \\
            \bm{A}p_N &= &\lambda_Np_N\\
        \end{array}
\end{equation*}

Next, it is shown how the eigenvalue equation shown above is related to the maximization of the variance of the data. For this purpose, the single vector $\bm{b}_1 \in \mathbb{R}^D$ is considered only, which is the first column of the matrix $\bm{B}$ and therefore the first of $M$ orthonormal basis vectors. By exploiting the i.i.d. assumption of the data, the first component of $\bm{z}_n \in \mathbb{R}^M$ of a single data point $\bm{x}_n \in \mathbb{R}^D$ is given by

\begin{equation}\label{eq:single_z}
    z_{1n} = \bm{b}_1^\top\bm{x}_n.
\end{equation}

Thus, $z_{1n}$ is the coordinate of the orthogonal projection of $\bm{x}_n$ onto the one-dimensional subspace spanned by $\bm{b}_1$. Then, the variance of $z_{1n}$ of $\bm{z} \in \mathbb{R}^M$, that is maximized by $\bm{b}_1$, is defined as

\begin{equation}\label{eq:pca_var}
    \mathbb{V}_1:= \mathbb{V}[z_1] = \frac{1}{N}\sum_{n=1}^N z_{1n}^2.
\end{equation}

The relation between the variance and the factorized matrix in Theorem \ref{th:eigendecomposition} becomes evident by substituting equation (\ref{eq:single_z}) into equation (\ref{eq:pca_var}) as

\begin{equation}\label{eq:pca_variance_objective}
    \begin{aligned}
        \mathbb{V}_1 &= \frac{1}{N}\sum\limits^N_{n=1}(\bm{b}_1^\top\bm{x}_n)^2 \\
        &= \frac{1}{N}\sum\limits^N_{n=1} \bm{b}_1^\top \bm{x}_n \bm{x}_n^\top \bm{b}_1 \\
        &= \bm{b}_1^\top(\frac{1}{N}\sum\limits^N_{n=1} \bm{x}_n \bm{x}_n^\top ) \bm{b}_1 \\
        &= \bm{b}_1^\top \bm{\Sigma} \bm{b}_1.
    \end{aligned}
\end{equation}

Maximizing the variance defined in (\ref{eq:pca_variance_objective}) results in finding vector $\bm{b}_1$. But arbitrarily increasing the magnitude of $\bm{b}_1$ increases $\mathbb{V}_1$, which is why restricting all solutions to unit vector size is necessary:

\begin{equation*}
    || \bm{b}_1 || ^2 = 1 \Leftrightarrow || \bm{b}_1 || = 1.
\end{equation*}

Restricting the solution space results into a constrained optimization problem, given as
\begin{align*}
    \mathrm{max} \text{ } \bm{b}_1^\top \bm{\Sigma} \bm{b}_1 \\
    \text{s.t. } || \bm{b}_1 ||^2 = 1.
\end{align*}

Applying the method of Lagrange multipliers, the new objective function is obtained as

\begin{equation*}
    \mathcal{L}(\bm{b}_1, \lambda) = \bm{b}_1^\top \bm{\Sigma} \bm{b}_1 + \lambda_1(1-\bm{b}_1^\top\bm{b}_1).
\end{equation*}

The partial derivative of $\mathcal{L}$ with respect to $\bm{b}_1$ gives

\begin{align*}
    \frac{\partial{\mathcal{L}}}{\partial{\bm{b}_1}} &= 2\bm{b}_1^\top \bm{\Sigma} - \lambda_1 \bm{b}_1^\top \\
    0 &= 2\bm{b}_1^\top \bm{\Sigma} - \lambda_1 \bm{b}_1^\top \\
    \bm{\Sigma} \bm{b}_1 &= \lambda_1 \bm{b}_1.
\end{align*}

By the definition shown above, $\bm{b}_1$ has to be an eigenvector of $\bm{\Sigma}$ with the lagrangian multiplier as corresponding eigenvalue. Then, computing the derivative with respect to $\lambda_1$ results in
\begin{align*}
    \frac{\partial{\mathcal{L}}}{\partial{\lambda_1}} &= 1 - \bm{b}_1^\top\bm{b}_1 \\
    0 &= 1 - \bm{b}_1^\top\bm{b}_1 \\
    \bm{b}_1^\top\bm{b}_1 &= 1.
\end{align*}

Finally, the variance objective from equation (\ref{eq:pca_variance_objective}) can be rewritten as 
\begin{equation}
    \mathbb{V}_1 = \bm{b}_1^\top \bm{C} \bm{b}_1 = \lambda_1 \bm{b}_1^\top \bm{b}_1 = \lambda_1. 
\end{equation}

The variance of the projected data equals the eigenvalue that is related to the basis vector $\bm{b}_1$, which spans the subspace the data is projected onto. In other words, the eigenvector that is associated with the biggest eigenvalue $\lambda_1$ is the first principal component.

Since the vector $\bm{b}_1^\top$ is considered a linear transformation, computing its inverse projects $z_{1n}$ back into the original data space $\mathbb{R}^D$, such that

\begin{equation*}
    \tilde{\bm{x}}_n = \bm{b}_1z_{1n} = \bm{b}_1 \bm{b}_1^\top \bm{x}_n.
\end{equation*}

So far, only the solution for the first principal component has been derived. Generalizing this schema onto a $M$-dimensional subspace with maximal variance involves some way of mathematical induction. The idea is to subtract the effect of the first $M-1$ principal components $\bm{b}_1, \dots, \bm{b}_{M-1}$ from the data and thereby trying to find the principal components that compress the remaining information. For a detailed investigation of this general case, reference is made to \cite[pp.291]{dei_2020}


\end{document}