\documentclass[../../../main.tex]{subfiles}
\begin{document}

\subsection{Intrusion Detection Systems} \label{subsec:intrusion_detection_systems}
Classic security mechanisms, such as encryption or firewalls, are considered as preventive measures for protecting IT infrastructures. However, in order to be able to react to security breaches that have already occurred, additional reactive mechanisms are required. To complement preventive measures, \glspl{ids} have been commercially available since the late 1990s \cite[27]{whitman_principles_2018}. Generally, the main reason for operating an \gls{ids} is to monitor and analyze computer networks or systems in order to identify anomalies, intrusions or privacy violations \cite{hindy2020taxonomy}. Specifically, the following three significant advantages are pointed out \cite[391]{whitman_principles_2018}.

\begin{itemize}
    \item \glspl{ids} can detect the preliminaries of attacks, in particular the organized gathering of information about networks and defense mechanisms (attack reconnaissance), and thus enable the prevention or mitigation of damage to information assets.
    \item \glspl{ids} can help protect information assets when known vulnerabilities cannot be fixed fast enough, notably in the context of an rapidly changing threat environment.
    \item The occurrence of unknown security vulnerabilities (zero day vulnerabilities) is not predictable, meaning that no specific preparations can be made for them. However, \glspl{ids} can identify processes in the IT system that deviate from the normal state and thus contribute to the detection of zero day attacks.
\end{itemize}

For an effective \gls{ids}, it is important to be able to detect as many steps as possible within the prototypical attack sequence, also called kill chain \cite[393]{whitman_principles_2018}. Since a successful intrusion into a system can be stopped at several points in this sequence, the effectiveness of the \gls{ids} increases with its functionality. The following categorization (see Figure \ref{fig:intrusion-taxonomy}) of intrusion attempts according to \cite{kendall1999database} reflects parts of that chain.

\begin{figure}[t]
    \centering
    \includestandalone{2_mainmatter/2_preliminaries/1_intrusion_detection/tikz/intrusion_taxonomy}
    \caption{A categorization of intrusions into different subtypes}
    \label{fig:intrusion-taxonomy}
\end{figure}

Probing refers to the preambles of actual attacks, also known as attack reconnaissance. This includes obtaining information about an organization and its network behavior (footprinting) and obtaining detailed information about the used operating systems, network protocols or hardware devices (fingerprinting). 
\newpage
\gls{dos} refers to an attack aimed at disabling a particular service for legitimate users by overloading the target systems processing capacity. \gls{r2l} attacks attempt to gain local access to the target system via a network connection. And one step further, \gls{u2r} attacks start out with user access on the system and gain root access by exploiting vulnerabilities. Individual intrusion types can but do not have to be executed consecutively. For example, it is not mandatory to render a system incapable of action by \gls{dos} in order to subsequently gain local access to the target. However, this may be part of a strategy.

\begin{figure}[b]
    \centering
    \includestandalone{2_mainmatter/2_preliminaries/1_intrusion_detection/tikz/ids_taxonomy}
    \caption{A taxonomy for \gls{ids} that distincts systems based on the utilized detection method or the  scope in which it operates}
    \label{fig:ids-taxonomy}
\end{figure}

Additionally, \glspl{ids} are generally categorized by the detection scope and the employed attack detection method \cite{milenkoski2015evaluating}. Hence, a distinction is made between \glspl{hids} and \glspl{nids}. While a \gls{hids} resides on a system, known as host, only monitoring local activities, a \gls{nids} resides on a network segment and monitors remote attacks that are carried out across the segment. In general, \gls{hids} are advantageous if individual hosts are to be monitored. \gls{nids}, on the other hand, are able to monitor traffic coming in and out of several hosts simultaneously. The disadvantage of \gls{nids}, however, is that attacks can only be detected if they are also reflected in the network traffic. Pioneering examples in the context of \gls{hids} are the Intrusion-Detection Expert System (IDES) \cite{lunt1992real} and the Multics Intrusion Detection and Alerting System (MIDAS) \cite{sebring1988expert}. A prominent representative in the area of \gls{nids} is NetSTAT \cite{vigna1998netstat} \cite{vigna1999netstat}.

Furthermore, \glspl{ids} are categorized into misuse-based detection and anomaly-based detection (see Figure \ref{fig:ids-taxonomy}). A misuse-based approach defines a model that describes intrusive behaviour and compares the system or network state against that model. An anomaly-based \gls{ids}, on the other hand, creates a statistical baseline profile of the system’s or network’s normal state and compares it with monitored activities. The concept of the \gls{nids} origins from \cite{denning1987intrusion}. 

Obviously, these two opposing approaches offer different advantages and disadvantages and are capable of complementing each other. Misuse-based systems can detect malicious behavior with a low false positive rate, but assume that the exact patterns  are known. Typically, this method cannot detect novel \cite[403]{whitman_principles_2018}, metamorphic or polymorphic attacks \cite[236]{szor2005art}. An anomaly-based approach allows both known and unknown attacks to be detected, but the frequent occurrence of false-positive estimations is a major challenge. Given the amount of data that occurs in computer systems and networks nowadays, the generation of false alarms for even a fraction of this amount can render the \gls{ids} operationally unusable, since no network administrator can investigate such a large number of incidents in detail \cite{axelsson2000base}. Most implementations of misuse-based systems rely on the creation of signatures. A signature is similar to a collection of rules that describes an attack pattern. The most prominent signature-based \gls{ids} is Snort \cite{roesch1999snort}. The payload-based anomaly detector PAYL \cite{wang2004anomalous} is an example for an anomaly-based system. Hybrid approaches are conceivable in order to circumvent the respective disadvantages of different subcategories of \glspl{ids}. The interested reader is referred to the following works \cite{DEPREN2005713} \cite{zhang2006hybrid} \cite{beer_hybrid2021}.
\end{document}