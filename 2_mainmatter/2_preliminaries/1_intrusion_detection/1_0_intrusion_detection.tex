\documentclass[../../../main.tex]{subfiles}
\begin{document}
\section{Intrusion Detection}\label{sec:intrusion_detection}

This section introduces the topic of intrusion detection and some important subcategories in the context of the thesis. First, \glspl{ids} are described and categorized according to different characteristics. The advantages and disadvantages of individual types of \acrshort{ids} are explained and examples are referred to individually. The topic of network flow monitoring is then addressed, since it is an essential component of network monitoring that is incorporated in the proposed architecture. The role of network flows in the context of network monitoring and management is motivated and the typical processes of a flow exporter are described. In the context of the shortcomings of conventional \glspl{ids} for protecting large scale systems against coordinated and distributed attacks \glspl{cids} are introduced. For this purpose, it is first described what coordinated and distributed attacks are. Then it is explained what constitutes a \gls{cids}, what requirements exist for the realization of \gls{cids}, and what components and processes \acrshort{cids} typically consist of.

\subfile{1_1_intrusion_detection_systems}

\subfile{1_2_network_flow_monitoring}

\subfile{1_3_collaborative_intrusion_detection}

\end{document}