\documentclass[../../../main.tex]{subfiles}
\begin{document}
\section{Intrusion Detection}

This section introduces the topic of intrusion detection and some important subcategories in the context of the thesis. First, IDSs are described and categorized according to different characteristics. The advantages and disadvantages of individual types of IDS are explained and examples are referred to individually. The topic of network flow monitoring is then addressed, since it is an essential component of network monitoring that is incorporated in the proposed architecture. The role of network flows in the context of network management is motivated and the typical processes of a flow exporter are described. In the context of the shortcomings of conventional IDSs for protecting large scale systems against coordinated and distributed attacks, Collaborative Intrusion Detection Systems (CIDSs) are introduced. For this purpose, it is first described what coordinated and distributed attacks are.  Then it is explained what constitutes CIDS, what requirements exist for the realization of CIDS, and what components and processes CIDS typically consist of.

\subfile{1_1_intrusion_detection_systems}

\subfile{1_2_network_flow_monitoring}

\subfile{1_3_collaborative_intrusion_detection}

\end{document}