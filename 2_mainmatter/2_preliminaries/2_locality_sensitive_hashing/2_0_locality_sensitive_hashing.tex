\documentclass[../../../main.tex]{subfiles}
\begin{document}
\section{Locality Sensitive Hashing}
Finding similar objects, formally known as the nearest-neighbour search problem, is a central problem in computer science in general. And besides the exact search, it is also an important topic in the field of intrusion detection. For example, exact searches allow to compare the signatures of known malware with low false positive rates. Considering polymorphic attacks, however, it is essential to also detect all objects that are similar to the known attack and thus detect modified forms of it. In this context, this section presents a well-studied approach called \textit{Locality Sensitive Hashing} (LSH) that solves an approximate version of the nearest-neighbour search problem by partitioning the search space with a hash function. This way, the searching problem is reduced to pairs that are most likely to be similar. Besides the application for solving the NN search problem, for which LSH was originally conceived, the concept has been proven to be effective for numerous other use cases, such as dimensionality reduction, clustering or classification. As the proposed generative pattern database integrates a locality-sensitive hash function for enabling both similarity search and data parallelism, LSH and a specific variant called \textit{Random Projection} (RP) is described in detail in this section.

Section \ref{subsec:approximate_nearest_neighbour_problem} introduces the approximate version of the nearest neighbour search problem as it is a prerequisite for the general definition of LSH. After that, Section \ref{subsec:locality-sensitive-hashes} begins by clarifying the core idea of LSH and subsequently explains how to construct a locality-sensitive hash function. Lastly, a specific family of LSH that uses the cosine distance for similarity calculations, also known as \textit{Random Projection} is presented in Section \ref{subsec:random_projection}.

\subfile{2_1_nearest_neighbour_problem}

\subfile{2_2_locality_sensitive_hash_functions}

\subfile{2_3_random_projection}

\end{document}