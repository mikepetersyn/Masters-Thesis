\documentclass[tikz]{standalone}
\usetikzlibrary{arrows.meta,positioning,calc}
\usepackage{pgfplots, pgfplotstable}
\pgfplotsset{compat=1.17}
\begin{document}
  \def\p{0.0}%
  \def\centerx{0}%
  \def\centery{0}%
  \pgfplotsset{
      colormap={cool}{rgb255(0cm)=(255,255,255); rgb255(1cm)=(0,128,255); rgb255(2cm)=(255,0,255)}
  }%
  \begin{tikzpicture}
      \begin{axis}[
        width=5cm,height=5cm,
        xlabel=$x_1$,
        ylabel=$x_2$,
        xlabel shift={-10pt},
        ylabel shift={-10pt},
        zmin=0,
        zmax=0.2,
        domain=-4:4,
        enlarge x limits,
        enlarge y limits,
        scaled y ticks=false,
        scaled x ticks=false,
        xtick={-3, 0, 3},
        xticklabels={$-3$, $0$, $3$},
        ytick={-3, 0, 3},
        yticklabels={$-3$, $0$, $3$},
        zticklabels={},
        view={45}{45},
        % colorbar horizontal,
        colormap name=cool,
        % colorbar style={
        %   xtick={0, 0.05, 0.1, 0.15, 0.2},
        %   xticklabels={$0$, $0.05$, $0.1$, $0.15$, $0.2$},
        % }
        ]%
        \addplot3[surf, samples=40, shader=faceted,thin]
        {1/(2 *pi* sqrt(1-\p^2))* exp(-((x-\centerx)^2+(y-\centery)^2-2*\p*(x-\centerx)*(y-\centery))/(2*(1-\p^2))};%
      \end{axis}
  \end{tikzpicture}
\end{document}