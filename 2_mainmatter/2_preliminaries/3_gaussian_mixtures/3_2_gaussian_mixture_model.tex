\documentclass[../../../main.tex]{subfiles}
\begin{document}
\subsection{The Gaussian Mixture Model} \label{subsec:gaussian_mixture_model}

Using statistical estimation techniques, it is possible to estimate an unobservable underlying probability density function from observed data. This allows data to be compactly represented with a density from a parametric family, such as a Gaussian distribution. However, all conventional parametric distributions are limited in their modeling capabilities when confronted with real data. For example, considering data that follows a multimodal distribution, i.e., have more than one center, a density estimate using a simple Gaussian distribution is not sufficient to effectively represent data distribution.

The idea is to represent a multimodal distribution by constructing a linear combination of multiple simple distributions, each of which representing a unimodal sub-population of the data, which is formalized under the term \textit{mixture model} \cite[p.111]{bis_2006}.

\begin{definition}[Mixture Model] \cite[p. 315]{dei_2020}\label{th:mixture_model}
A mixture model is a linear combination of $K$ parametric distributions $p_k$, each weighted by a mixture weight $\pi_k$, with the following form 
\begin{align*}
    &p(x) = \sum\limits_{k=1}^K \pi_k p_k (x), \\
    &0 \leq \pi_k \leq 1, \sum\limits_{k=1}^K \pi_k = 1.
\end{align*}
\end{definition}

A distribution $p_k$ within this model is called mixture component and the sum of the mixture weights equals to 1, such that the probability density of the mixture components equals to 1 as well. 

Mixture Models can use any arbitrary parametric distribution as component density, but the most common mixture model is the Gaussian mixture model, using Gaussians as components \cite[214]{has_2009}. First, \acrshort{gmm}s utilize the practical mathematical properties of Gaussians, introduced earlier in Section~\ref{subsec:gaussian_distribution}. 

Furthermore, the quality of the model, in terms of its ability to estimate real data distributions, is theoretically supported by the Central Limit Theorem (see Definition~\ref{th:central_limit}), which states, that most data distributions converge to a normal distribution on average as the number of data points increases \cite[p.222]{jay_2003}.

\begin{theorem}[Central Limit Theorem]\label{th:central_limit} \cite[p.241]{montgomery_2010}
Consider $N$ i.i.d. random variables $X_i$ with $\mathbb{E}[X_i]=\mu$ and $\mathbb{V}[X_i]=\sigma^2$ and let $S_N=\sum^N_{i=1}X_i$. It can be shown that, as $N$ increases, that
\begin{equation*}
    p(S_N) \sim \mathcal{N}\left(\mu, \frac{\sigma^2}{N}\right).
\end{equation*}
\end{theorem}

By further specifying the general definition of mixture models (see Definition~\ref{th:mixture_model}), we obtain the definition of a \gls{gmm} (see Definition~\ref{th:gaussian_mixture_model}). Instead of using an arbitrary parametric distribution as the component density, the Gaussian distribtion is utilized. Furthermore, the set of parameters $\bm{\theta}$ of the \gls{gmm} is introduced.

\begin{definition}[Gaussian Mixture Model]\label{def:gmm} \cite[p. 315]{dei_2020}\label{th:gaussian_mixture_model}
A \acrlong{gmm} is a combination of a finite number of $K$ Gaussian distributions $\mathcal{N}(\bm{x}|\theta_k)$ which is fully described by a probability density function $p$ and its parameter set $\bm{\theta}$ as
\begin{equation}\label{eq:gmm_def}
    \begin{aligned}
        &p(\bm{x} \, | \, \bm{\theta}) = \sum\limits_{k=1}^K \pi_k \, \mathcal{N}(\bm{x} \, | \, \theta_k), \\
        &0 \leq \pi_k \leq 1, \sum\limits_{k=1}^K \pi_k = 1, \\
        &\bm{\theta} := \{\bm{\mu}_k, \bm{\Sigma}_k, \pi_k \, | \, k = 1, \dots, K \}.
    \end{aligned}
\end{equation}
\end{definition}




\end{document}