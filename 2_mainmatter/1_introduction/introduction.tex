\documentclass[../../main.tex]{subfiles}
\begin{document}

\chapter{Introduction}

\section{Problem Description}

\section{Contribution}

\section{Thesis Outline}

The thesis consists of a total of 6 chapters and is mainly organized into a theoretical part (Chapter~\ref{ch:preliminaries} and \ref{ch:related_work}) and a practical part (Chapter~\ref{ch:generative_pattern_database} and \ref{ch:experimental_evaluation}). The theoretical part exclusively covers contents that can be sourced in the literature. In contrast, the practical part develops original contents, some of which are based on knowledge and concepts acquired in the theoretical part. A detailed overview of the thesis structure is given in the following.

\begin{description}
    \item[Chapter~\ref{ch:preliminaries}] First, the preliminaries chapter provides the reader with basic background information regarding the topics of Intrusion Detection, Locality Sensitive Hashing, Gaussian Mixtures and Principal Component Analysis. While the topic Intrusion Detection gives a general overview of the context of this work, the remaining topics are more specific and provide a detailed insight into techniques and algorithms that are used to realize the proposed \acrshort{cids} architecture.
    \item[Chapter~\ref{ch:related_work}] In this chapter, the current state of the art in three different areas related to this thesis is summarized and briefly discussed. First, techniques for data dissemination in \glspl{cids} are examined and the main challenges in this context are highlighted. Then, various established and novel approaches that use some form of similarity hashing in the context of intrusion detection are presented. Finally, the advantages of generative algorithms for intrusion detection are analyzed.
    \item[Chapter~\ref{ch:generative_pattern_database}] Based on the main challenges for data dissemination, this chapter develops a \gls{cids} architecture that realizes the principles of Generative Pattern Dissemination, introducing a novel approach in the field of intrusion detection. After a general introduction of the idea and functionality, a detailed specification of the architecture and the developed algorithms is given.
    \item[Chapter~\ref{ch:experimental_evaluation}]
    \item[Chapter~\ref{ch:conclusion}]    
\end{description}

\end{document}