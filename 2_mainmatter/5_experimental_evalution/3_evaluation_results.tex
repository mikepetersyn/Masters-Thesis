\documentclass[../../main.tex]{subfiles}
\begin{document}

\section{Evaluation Results}

This section presents the results from the experiments described in Section~\ref{sec:experimental_setup}. First, the results from the baseline experiments are outlined in Section~\ref{sec:baseline}. Then, in Section~\ref{sec:classification_improvement} the experimental results regarding the detection performance of the presented \gls{cids} architecture are presented. Lastly, the results in the context of overhead reduction are shown in Section~\ref{sec:overhead_reduction}.

\subsection{Baseline Experiments}\label{sec:baseline}

The tables regarding the classification results in the next two sections are interpreted as follows. The columns, which are grouped into skipped classes refer to the class that is removed from the respective training dataset that is indicated by the first row dimension. Four metrics are computed in a $k$-class problem setting for each iteration. First, the balanced accuracy score is evaluated. It shows the summarized detection performance over all $K$ classes and is computed from the diagonal elements and the row sums of the confusion matrix $\mathbf{C} = (c_{ij})$. Next, the precision, recall and the balanced F-score for the class that was removed from the training dataset are computed. Thus, while the accuracy provides a way to interpret the performance regarding all classes, the latter three metrics focus on the correct detection of the class that was removed from the local training dataset. Table~\ref{tab:missing_classes} shows the results from the experiments without any collaboration mechanism. As indicated by the accuracy, all classes, except for the removed class, are detected well. Note that no false positives regarding the removed class are introduced either, because the classifier did not include that class in training at all. Hence, all metrics that refer to the removed class consequently have a value of zero. In particular, by introducing data from other infrastructures in the training dataset, the classifier may not only perform better by detecting the removed class correctly, which is indicated by the recall. It may also be the case that other classes are falsely classified as the removed class, which is measured by the precision. Therefore, a low precision is interpreted as the introducion of noise that comes along with the data exchange in the \gls{cids}. Table~\ref{tab:baseline_detection} shows the results of the alternative \gls{cids} approach, where the original data is exchanged directly between members. These results are considered as a benchmark for the presented \gls{cids} architecture in terms of the detection performance. 
\begin{align*}
    \text{balanced accuracy} \quad &= \quad \frac{1}{K} \sum\limits_{k=1}^K \frac{c_{kk}}{\sum_{j=1}^K c_{kj}} \\
    \text{precision} \quad &= \quad \frac{TP}{TP + FP} \\
    \text{recall} \quad &= \quad \frac{TP}{TP + FN} \\
    \text{F1} \quad &= \quad 2 \cdot \frac{\text{precision} \cdot \text{recall}}{\text{precision} + \text{recall}}
\end{align*}
The overhead that is introduced by the direct exchange of original data between members is listed in Table~\ref{tab:num_samples_parameters}. The second column shows the number of attack samples directly after the flow export. The third column shows the number of samples after removing duplicates from the collection of attack samples. The fourth column lists the number of parameters, i.e. float values, that are to be exchanged, that is the product of attack flows and their respective dimensionality (see Table~\ref{tab:flow_features}).

\begin{table}[H]
    \footnotesize
    \centering
    \setlength{\extrarowheight}{0pt}
    \addtolength{\extrarowheight}{\aboverulesep}
    \addtolength{\extrarowheight}{\belowrulesep}
    \addtolength{\tabcolsep}{-0.4em}
    \setlength{\aboverulesep}{0pt}
    \setlength{\belowrulesep}{0pt}
    \setlength{\extrarowheight}{.1em}
    \begin{tabular}{clllllllll} 
    \toprule
    \multirow{3}{*}{\textbf{Dataset}} & \multicolumn{1}{c}{\multirow{3}{*}{\textbf{Metric}}} & \multicolumn{8}{c}{\textbf{Skipped Classes}}                                                                                                                                                                                                                                                                                                                    \\ 
    \cmidrule{3-10}
                                      & \multicolumn{1}{c}{}                                 & \multicolumn{2}{c}{Brute Force}                   & \multicolumn{2}{c}{Web Attack}                         & \multicolumn{2}{c}{DoS}                                                                                                                                    & \multicolumn{1}{c}{\multirow{2}{*}{DDoS}} & \multicolumn{1}{c}{\multirow{2}{*}{Bot}}  \\ 
    \cmidrule(lr){3-3}\cmidrule(lr){4-4}\cmidrule(lr){5-5}\cmidrule(lr){6-6}\cmidrule(lr){7-7}\cmidrule(lr){8-8}
                                      & \multicolumn{1}{c}{}                                 & \multicolumn{1}{c}{SSH} & \multicolumn{1}{c}{Web} & \multicolumn{1}{c}{SQL Inj.} & \multicolumn{1}{c}{XSS} & \multicolumn{1}{c}{\begin{tabular}[c]{@{}c@{}}HTTP\\ High Vol.\end{tabular}} & \multicolumn{1}{c}{\begin{tabular}[c]{@{}c@{}}HTTP \\Low Vol.\end{tabular}} & \multicolumn{1}{c}{}                      & \multicolumn{1}{c}{}                      \\ 
    \midrule
    \multirow{4}{*}{I}                & Acc.                                                 & 0.8677                  & 0.8684                  & 0.8886                       & 0.8785                  & 0.8679                                                                       & 0.8677                                                                      & 0.8778                                    & 0.8685                                    \\
                                      & Prec.                                                & 0.                      & 0.                      & 0.                           & 0.                      & 0.                                                                           & 0.                                                                          & 0.                                        & 0.                                        \\
                                      & Rec.                                                 & 0.                      & 0.                      & 0.                           & 0.                      & 0.                                                                           & 0.                                                                          & 0.                                        & 0.                                        \\
                                      & F1                                                   & 0.                      & 0.                      & 0.                           & 0.                      & 0.                                                                           & 0.                                                                          & 0.                                        & 0.                                        \\ 
    \midrule
    \multirow{4}{*}{II}               & Acc.                                                 & 0.8475                  & 0.8497                  & 0.8837                       & 0.8467                  & 0.8844                                                                       & 0.8473                                                                      & 0.8466                                    & 0.8472                                    \\
                                      & Prec.                                                & 0.                      & 0.                      & 0.                           & 0.                      & 0.                                                                           & 0.                                                                          & 0.                                        & 0.                                        \\
                                      & Rec.                                                 & 0.                      & 0.                      & 0.                           & 0.                      & 0.                                                                           & 0.                                                                          & 0.                                        & 0.                                        \\
                                      & F1                                                   & 0.                      & 0.                      & 0.                           & 0.                      & 0.                                                                           & 0.                                                                          & 0.                                        & 0.                                        \\ 
    \midrule
    \multirow{4}{*}{III}              & Acc.                                                 &                         &                         &                              &                         & 0.7456                                                                       & 0.7036                                                                      & 0.7018                                    &                                           \\
                                      & Prec.                                                &                         &                         &                              &                         & 0.                                                                           & 0.                                                                          & 0.                                        &                                           \\
                                      & Rec.l                                                &                         &                         &                              &                         & 0.                                                                           & 0.                                                                          & 0.                                        &                                           \\
                                      & F1                                                   &                         &                         &                              &                         & 0.                                                                           & 0.                                                                          & 0.                                        &                                           \\
    \bottomrule
    \end{tabular} 
    \caption[Baseline Detection Performance with Missing Class Information]{The table shows the classification performance for isolated \acrshortpl{ids} with knowledge gaps. No collaboration mechanism is applied.}
    \label{tab:missing_classes}
\end{table}

\begin{table}[H]
    \footnotesize
    \centering
    \setlength{\extrarowheight}{0pt}
    \addtolength{\extrarowheight}{\aboverulesep}
    \addtolength{\extrarowheight}{\belowrulesep}
    \addtolength{\tabcolsep}{-0.4em}
    \setlength{\aboverulesep}{0pt}
    \setlength{\belowrulesep}{0pt}
    \setlength{\extrarowheight}{.1em}
    \begin{tabular}{clllllllll} 
    \toprule
    \multirow{3}{*}{\textbf{Dataset}} & \multicolumn{1}{c}{\multirow{3}{*}{\textbf{Metric}}} & \multicolumn{8}{c}{\textbf{Removed Class}}                                                                                                                                                                                                                                                                                                                      \\ 
    \cmidrule{3-10}
                                      & \multicolumn{1}{c}{}                                 & \multicolumn{2}{c}{Brute Force}                   & \multicolumn{2}{c}{Web Attack}                         & \multicolumn{2}{c}{DoS}                                                                                                                                    & \multicolumn{1}{c}{\multirow{2}{*}{DDoS}} & \multicolumn{1}{c}{\multirow{2}{*}{Bot}}  \\ 
    \cmidrule(l){3-8}
                                      & \multicolumn{1}{c}{}                                 & \multicolumn{1}{c}{SSH} & \multicolumn{1}{c}{Web} & \multicolumn{1}{c}{SQL Inj.} & \multicolumn{1}{c}{XSS} & \multicolumn{1}{c}{\begin{tabular}[c]{@{}c@{}}HTTP\\ High Vol.\end{tabular}} & \multicolumn{1}{c}{\begin{tabular}[c]{@{}c@{}}HTTP \\Low Vol.\end{tabular}} & \multicolumn{1}{c}{}                      & \multicolumn{1}{c}{}                      \\ 
    \midrule
    \multirow{4}{*}{I}                & Acc.                                                 & 0.9783                  & 0.8769                  & 0.9087                       & 0.9643                  & 0.8659                                                                       & 0.9243                                                                      & 0.8474                                    & 0.8684                                    \\
                                      & Prec.                                                & 0.9999                  & 1.0                     & 1.0                          & 1.0                     & 0.9937                                                                       & 0.9986                                                                      & 0.0*                                      & 0.0*                                      \\
                                      & Rec.                                                 & 0.9955                  & 0.1677                  & 0.1818                       & 0.8636                  & 0.0674                                                                       & 0.5947                                                                      & 0.0                                       & 0.0                                       \\
                                      & F1                                                   & 0.9977                  & 0.2872                  & 0.3076                       & 0.9268                  & 0.1262                                                                       & 0.7454                                                                      & 0.0*                                      & 0.0*                                      \\ 
    \midrule
    \multirow{4}{*}{II}               & Acc.                                                 & 0.9551                  & 0.8445                  & 0.9206                       & 0.9191                  & 0.8906                                                                       & 0.9326                                                                      & 0.8465                                    & 0.8457                                    \\
                                      & Prec.                                                & 1.0                     & 0.0*                    & 1.0                          & 1.0                     & 0.9862                                                                       & 0.9982                                                                      & 0.0*                                      & 0.0*                                      \\
                                      & Rec.                                                 & 0.9813                  & 0.0                     & 0.3333                       & 0.6667                  & 0.0706                                                                       & 0.7825                                                                      & 0.0                                       & 0.0                                       \\
                                      & F1                                                   & 0.9905                  & 0.0*                    & 0.5                          & 0.8                     & 0.1317                                                                       & 0.8773                                                                      & 0.0*                                      & 0.0*                                      \\ 
    \midrule
    \multirow{4}{*}{III}              & Acc.                                                 &                         &                         &                              &                         & 0.7685                                                                       & 0.7316                                                                      & 0.7022                                    &                                           \\
                                      & Prec.                                                &                         &                         &                              &                         & 0.8653                                                                       & 0.9960                                                                      & 0.0*                                      &                                           \\
                                      & Recall                                               &                         &                         &                              &                         & 0.0918                                                                       & 0.1182                                                                      & 0.0                                       &                                           \\
                                      & F1                                                   &                         &                         &                              &                         & 0.1660                                                                       & 0.2114                                                                      & 0.0*                                      &                                           \\
    \bottomrule
    \end{tabular} 
    \caption[Baseline Detection Performance with Original Data Exchange]{The table shows the classification performance for \acrshortpl{ids} that exchange original data directly within a \gls{cids}.}
    \label{tab:baseline_detection}
\end{table}

\begin{table}[H]
    \footnotesize
    \centering
    \setlength{\extrarowheight}{0pt}
    \addtolength{\extrarowheight}{\aboverulesep}
    \addtolength{\extrarowheight}{\belowrulesep}
    \addtolength{\tabcolsep}{-0.4em}
    \setlength{\aboverulesep}{0pt}
    \setlength{\belowrulesep}{0pt}
    \setlength{\extrarowheight}{.1em}
    \begin{tabular}{crrr} 
    \toprule
    \textbf{Dataset } & \multicolumn{1}{c}{\begin{tabular}[c]{@{}c@{}}\textbf{Attack Samples}\\(before Deduplication)\end{tabular}} & \multicolumn{1}{c}{\begin{tabular}[c]{@{}c@{}}\textbf{Attack Samples}\\(after Deduplication)\end{tabular}} & \multicolumn{1}{c}{\begin{tabular}[c]{@{}c@{}}\textbf{Sum Parameters}\\($N \text{ samples} \cdot D \text{ dimensions}$)\end{tabular}}  \\ 
    \midrule
    I                 & \numprint{1623847}                                                                                                   & \numprint{356250}                                                                                                     & \numprint{13893750}                                     \\
    II                & \numprint{120152}                                                                                                     & \numprint{77757}                                                                                                      & \numprint{3032523}                                      \\
    III               & \numprint{43284}                                                                                                      & \numprint{7758}                                                                                                       & \numprint{302562}                                       \\
    \bottomrule
    \end{tabular} 
    \caption[Communication Overhead in the Baseline Experiment]{The table shows the communication overhead for directly exchanging original attack data between the members in the \gls{cids}.}
    \label{tab:num_samples_parameters}
\end{table}

\subsection{Classification Improvement}\label{sec:classification_improvement}

Table~\ref{tab:synthetic_classification} shows the results for the scenario, where members in the proposed \gls{cids} architecture enhance local datasets with synthetic data. Values that are equal ($\blacktriangleright$) or better ($\blacktriangle$) than the results from the baseline detection scenario, where original data is exchanged (see Table~\ref{tab:baseline_detection}), are displayed in bold font and marked with triangles respectively. Metric values from the baseline experiment that are marked with an asterisk are excluded from the comparison. Note that the asterisk marks metric values that are set to zero because of an ill-defined operation due to no predicted samples in the respective class. Table~\ref{tab:pattern_synthetic_classification} shows the results from the scenario where, in addition to synthetic data, the classification is supported by the usage of locality sensitive hashes, i.e., patterns. 

\begin{table}[H]
    \scriptsize
    \centering
    \setlength{\extrarowheight}{0pt}
    \addtolength{\extrarowheight}{\aboverulesep}
    \addtolength{\extrarowheight}{\belowrulesep}
    \addtolength{\tabcolsep}{-0.4em}
    \setlength{\aboverulesep}{0pt}
    \setlength{\belowrulesep}{0pt}
    \setlength{\extrarowheight}{.15em}
    \begin{tabular}{cclllllllll} 
    \toprule
    \multicolumn{1}{l}{\multirow{3}{*}{\begin{tabular}[c]{@{}l@{}}\textbf{Hash }\\\textbf{Size}\end{tabular}}} & \multirow{3}{*}{\textbf{Dataset}} & \multicolumn{1}{c}{\multirow{3}{*}{\textbf{Metric}}} & \multicolumn{8}{c}{\textbf{Skipped Classes}}                                                                                                                                                                                                                                                                                                                    \\ 
    \cmidrule{4-11}
    \multicolumn{1}{l}{}                                                                                       &                                   & \multicolumn{1}{c}{}                                 & \multicolumn{2}{c}{Brute Force}                   & \multicolumn{2}{c}{Web Attack}                         & \multicolumn{2}{c}{DoS}                                                                                                                                    & \multicolumn{1}{c}{\multirow{2}{*}{DDoS}} & \multicolumn{1}{c}{\multirow{2}{*}{Bot}}  \\ 
    \cmidrule(lr){4-4}\cmidrule(lr){5-5}\cmidrule(lr){6-6}\cmidrule(lr){7-7}\cmidrule(lr){8-8}\cmidrule(lr){9-9}
    \multicolumn{1}{l}{}                                                                                       &                                   & \multicolumn{1}{c}{}                                 & \multicolumn{1}{c}{SSH} & \multicolumn{1}{c}{Web} & \multicolumn{1}{c}{SQL Inj.} & \multicolumn{1}{c}{XSS} & \multicolumn{1}{c}{\begin{tabular}[c]{@{}c@{}}HTTP\\ High Vol.\end{tabular}} & \multicolumn{1}{c}{\begin{tabular}[c]{@{}c@{}}HTTP \\Low Vol.\end{tabular}} & \multicolumn{1}{c}{}                      & \multicolumn{1}{c}{}                      \\ 
    \midrule
    \multicolumn{1}{l}{\multirow{12}{*}{16}}                                                                   & \multirow{4}{*}{I}                & Acc.                                                 & 0.8534                  & 0.8943                  & 0.9290                       & 0.9643                  & 0.8637                                                                       & 0.8585                                                                      & 0.8566                                    & 0.8671                                    \\
    \multicolumn{1}{l}{}                                                                                       &                                   & Prec.                                                & 1.0                     & 1.0                     & 1.0                          & 1.0                     & 0.9973                                                                       & 1.0                                                                         & 0.0*                                      & 0.0*                                      \\
    \multicolumn{1}{l}{}                                                                                       &                                   & Rec.                                                 & 0.0471                  & 0.1428                  & 0.3636                       & 0.8636                  & 0.0531                                                                       & 0.0076                                                                      & 0.0                                       & 0.0                                       \\
    \multicolumn{1}{l}{}                                                                                       &                                   & F1                                                   & 0.0899                  & 0.25                    & 0.5333                       & 0.9268                  & 0.1009                                                                       & 0.0151                                                                      & 0.0*                                      & 0.0*                                      \\ 
    \cmidrule{2-11}
    \multicolumn{1}{l}{}                                                                                       & \multirow{4}{*}{II}               & Acc.                                                 & 0.9566                  & 0.8865                  & 0.9208                       & 0.9261                  & 0.9217                                                                       & 0.8621                                                                      & 0.8468                                    & 0.8475                                    \\
    \multicolumn{1}{l}{}                                                                                       &                                   & Prec.                                                & 1.0                     & 1.0                     & 1.0                          & 1.0                     & 1.0                                                                          & 0.9958                                                                      & 0.0                                       & 0.0*                                      \\
    \multicolumn{1}{l}{}                                                                                       &                                   & Rec.                                                 & 0.9826                  & 0.3783                  & 0.3333                       & 0.7142                  & 0.3343                                                                       & 0.1311                                                                      & 0.0                                       & 0.0                                       \\
    \multicolumn{1}{l}{}                                                                                       &                                   & F1                                                   & 0.9912                  & 0.5490                  & 0.5                          & 0.8333                  & 0.5011                                                                       & 0.2317                                                                      & 0.0                                       & 0.0*                                      \\ 
    \cmidrule{2-11}
    \multicolumn{1}{l}{}                                                                                       & \multirow{4}{*}{III}              & Acc.                                                 &                         &                         &                              &                         & 0.7706                                                                       & 0.7304                                                                      & 0.7029                                    &                                           \\
    \multicolumn{1}{l}{}                                                                                       &                                   & Prec.                                                &                         &                         &                              &                         & 1.0                                                                          & 0.0                                                                         & 0.0*                                      &                                           \\
    \multicolumn{1}{l}{}                                                                                       &                                   & Rec.                                                 &                         &                         &                              &                         & 0.1                                                                          & 0.1075                                                                      & 0.0                                       &                                           \\
    \multicolumn{1}{l}{}                                                                                       &                                   & F1                                                   &                         &                         &                              &                         & 0.1818                                                                       & 0.1942                                                                      & 0.0*                                      &                                           \\ 
    \midrule
    \multirow{12}{*}{32}                                                                                       & \multirow{4}{*}{I}                & Acc.                                                 & 0.8576                  & 0.8672                  & 0.8886                       & 0.8667                  & 0.8639                                                                       & 0.8686                                                                      & 0.8667                                    & 0.8576                                    \\
                                                                                                               &                                   & Prec.                                                & 0.0*                    & 1.0                     & 0.0*                         & 1.0                     & 0.9942                                                                       & 1.0                                                                         & 0.0*                                      & 0.0*                                      \\
                                                                                                               &                                   & Rec.                                                 & 0.0                     & 0.0807                  & 0.0                          & 0.0818                  & 0.0490                                                                       & 0.0021                                                                      & 0.0                                       & 0.0                                       \\
                                                                                                               &                                   & F1                                                   & 0.0*                    & 0.1494                  & 0.0*                         & 0.1512                  & 0.0934                                                                       & 0.0043                                                                      & 0.0*                                      & 0.0*                                      \\ 
    \cmidrule{2-11}
                                                                                                               & \multirow{4}{*}{II}               & Acc.                                                 & 0.9420                  & 0.8573                  & 0.9210                       & 0.8468                  & 0.8615                                                                       & 0.8490                                                                      & 0.8465                                    & 0.8473                                    \\
                                                                                                               &                                   & Prec.                                                & 1.0                     & 1.0                     & 1.0                          & 0.0*                    & 1.0                                                                          & 0.9822                                                                      & 0.0                                       & 0.0*                                      \\
                                                                                                               &                                   & Rec.                                                 & 0.8495                  & 0.0675                  & 0.3333                       & 0.0                     & 0.1259                                                                       & 0.0607                                                                      & 0.0                                       & 0.0                                       \\
                                                                                                               &                                   & F1                                                   & 0.9186                  & 0.1265                  & 0.5                          & 0.0*                    & 0.2236                                                                       & 0.1144                                                                      & 0.0                                       & 0.0*                                      \\ 
    \cmidrule{2-11}
                                                                                                               & \multirow{4}{*}{III}              & Acc.                                                 &                         &                         &                              &                         & 0.7661                                                                       & 0.7254                                                                      & 0.7033                                    &                                           \\
                                                                                                               &                                   & Prec.                                                &                         &                         &                              &                         & 1.0                                                                          & 1.0                                                                         & 0.0*                                      &                                           \\
                                                                                                               &                                   & Rec.                                                 &                         &                         &                              &                         & 0.0816                                                                       & 0.0811                                                                      & 0.0                                       &                                           \\
                                                                                                               &                                   & F1                                                   &                         &                         &                              &                         & 0.1509                                                                       & 0.1501                                                                      & 0.0*                                      &                                           \\ 
    \midrule
    \multirow{12}{*}{48}                                                                                       & \multirow{4}{*}{I}                & Acc.                                                 & 0.8583                  & 0.8583                  & 0.8879                       & 0.8677                  & 0.8644                                                                       & 0.8583*                                                                     & 0.8566                                    & 0.8576                                    \\
                                                                                                               &                                   & Prec.                                                & 0.0*                    & 0.0*                    & 0.0*                         & 0.0*                    & 0.9953                                                                       & 0.*                                                                         & 0.0*                                      & 0.0*                                      \\
                                                                                                               &                                   & Rec.                                                 & 0.0                     & 0.0                     & 0.0                          & 0.0                     & 0.0599                                                                       & 0.0                                                                         & 0.0                                       & 0.0                                       \\
                                                                                                               &                                   & F1                                                   & 0.0*                    & 0.0*                    & 0.0*                         & 0.0*                    & 0.1130                                                                       & 0.*                                                                         & 0.0*                                      & 0.0*                                      \\ 
    \cmidrule{2-11}
                                                                                                               & \multirow{4}{*}{II}               & Acc.                                                 & 0.9494                  & 0.8499                  & 0.8839                       & 0.8470                  & 0.9213                                                                       & 0.8605                                                                      & 0.8470                                    & 0.8475                                    \\
                                                                                                               &                                   & Prec.                                                & 1.0                     & 0.0*                    & 0.0*                         & 0.0*                    & 1.0                                                                          & 0.9929*                                                                     & 0.0*                                      & 0.0*                                      \\
                                                                                                               &                                   & Rec.                                                 & 0.9161                  & 0.0                     & 0.0                          & 0.0                     & 0.3323                                                                       & 0.1163                                                                      & 0.0                                       & 0.0                                       \\
                                                                                                               &                                   & F1                                                   & 0.9562                  & 0.0*                    & 0.0*                         & 0.0*                    & 0.4988                                                                       & 0.2082*                                                                     & 0.0*                                      & 0.0*                                      \\ 
    \cmidrule{2-11}
                                                                                                               & \multirow{4}{*}{III}              & Acc.                                                 &                         &                         &                              &                         & 0.7636                                                                       & 0.7123                                                                      & 0.7029                                    &                                           \\
                                                                                                               &                                   & Prec.                                                &                         &                         &                              &                         & 1.0                                                                          & 1.0                                                                         & 0.0*                                      &                                           \\
                                                                                                               &                                   & Rec.                                                 &                         &                         &                              &                         & 0.0714                                                                       & 0.0349                                                                      & 0.0                                       &                                           \\
                                                                                                               &                                   & F1                                                   &                         &                         &                              &                         & 0.1333                                                                       & 0.0675                                                                      & 0.0*                                      &                                           \\
    \bottomrule
    \end{tabular} 
    \caption[Detection Performance with Synthetic Data]{The table shows the classification performance for \acrshortpl{ids} that participate in the proposed \gls{cids} and enhance their local training datasets with synthetic data.}
    \label{tab:synthetic_classification}
\end{table}

\begin{table}[H]
    \scriptsize
    \centering
    \setlength{\extrarowheight}{0pt}
    \addtolength{\extrarowheight}{\aboverulesep}
    \addtolength{\extrarowheight}{\belowrulesep}
    \addtolength{\tabcolsep}{-0.4em}
    \setlength{\aboverulesep}{0pt}
    \setlength{\belowrulesep}{0pt}
    \setlength{\extrarowheight}{.15em}
    \begin{tabular}{cclllllllll} 
    \toprule
    \multirow{3}{*}{\begin{tabular}[c]{@{}c@{}}\textbf{Hash }\\\textbf{Size}\end{tabular}} & \multirow{3}{*}{\textbf{Dataset}} & \multicolumn{1}{c}{\multirow{3}{*}{\textbf{Metric}}} & \multicolumn{8}{c}{\textbf{Removed Class}}                                                                                                                                                                                                                                                                                                                    \\ 
    \cmidrule{4-11}
                                                                                           &                                   & \multicolumn{1}{c}{}                                 & \multicolumn{2}{c}{Brute Force}                   & \multicolumn{2}{c}{Web Attack}                         & \multicolumn{2}{c}{DoS}                                                                                                                                    & \multicolumn{1}{c}{\multirow{2}{*}{DDoS}} & \multicolumn{1}{c}{\multirow{2}{*}{Bot}}  \\ 
    \cmidrule(l){4-5}\cmidrule(l){6-7}\cmidrule(l){8-9}
                                                                                           &                                   & \multicolumn{1}{c}{}                                 & \multicolumn{1}{c}{SSH} & \multicolumn{1}{c}{Web} & \multicolumn{1}{c}{SQL Inj.} & \multicolumn{1}{c}{XSS} & \multicolumn{1}{c}{\begin{tabular}[c]{@{}c@{}}HTTP\\ High Vol.\end{tabular}} & \multicolumn{1}{c}{\begin{tabular}[c]{@{}c@{}}HTTP \\Low Vol.\end{tabular}} & \multicolumn{1}{c}{}                      & \multicolumn{1}{c}{}                      \\ 
    \midrule
    \multirow{12}{*}{16}                                                                   & \multirow{4}{*}{I}                & Acc.                                                 & 0.7763                  & 0.8103                  & 0.8394                       & 0.8619                  & 0.7770                                                                       & 0.7735                                                                      & \textbf{0.8562}$\blacktriangle$                          & \textbf{0.7912}$\blacktriangle$                          \\
                                                                                           &                                   & Prec.                                                & \textbf{1.0}$\blacktriangle$           & \textbf{1.0}$\blacktriangle$           & \textbf{1.0}$\blacktriangle$                & \textbf{1.0}$\blacktriangle$           & 0.9147                                                                       & 0.0                                                                         & 1.0                                       & 0.0*                                      \\
                                                                                           &                                   & Rec.                                                 & 0.0413                  & 0.1180                  & \textbf{0.3636}$\blacktriangle$             & 0.7272                  & 0.0455                                                                       & 0.0071                                                                      & 0.0001                                    & 0.0                                       \\
                                                                                           &                                   & F1                                                   & 0.0794                  & 0.2111                  & \textbf{0.5333}$\blacktriangle$             & 0.8421                  & 0.0867                                                                       & 0.0142                                                                      & 0.0002                                    & 0.0*                                      \\ 
    \cmidrule{2-11}
                                                                                           & \multirow{4}{*}{II}               & Acc.                                                 & 0.9536                  & 0.8848                  & 0.9179                       & \textbf{0.9216$\blacktriangle$}        & \textbf{0.9206}$\blacktriangleright$                                                              & 0.8611                                                                      & \textbf{0.8465}$\blacktriangleright$                          & 0.8445                                    \\
                                                                                           &                                   & Prec.                                                & \textbf{1.0}$\blacktriangleright$            & 1.0                     & \textbf{1.0}$\blacktriangleright$                 & \textbf{1.0}$\blacktriangleright$            & \textbf{0.9955}$\blacktriangleright$                                                              & \textbf{0.9916}$\blacktriangle$                                                            & 0.0                                       & 0.0*                                      \\
                                                                                           &                                   & Rec.                                                 & 0.9800                  & 0.3783                  & \textbf{0.3333}$\blacktriangleright$              & \textbf{0.7142}$\blacktriangle$        & \textbf{0.3333}$\blacktriangle$                                                             & 0.1308                                                                      & 0.0                                       & 0.0                                       \\
                                                                                           &                                   & F1                                                   & 0.9899                  & 0.5490                  & \textbf{0.5}$\blacktriangleright$                 & \textbf{0.8333}$\blacktriangle$        & \textbf{0.4994}$\blacktriangle$                                                             & 0.2312                                                                      & 0.0                                       & 0.0*                                      \\ 
    \cmidrule{2-11}
                                                                                           & \multirow{4}{*}{III}              & Acc.                                                 &                         &                         &                              &                         & 0.7666                                                                       & 0.7297                                                                      & 0.6961                                    &                                           \\
                                                                                           &                                   & Prec.                                                &                         &                         &                              &                         & \textbf{0.9245}$\blacktriangle$                                                             & \textbf{1.0}$\blacktriangle$                                                                & 0.3333                                    &                                           \\
                                                                                           &                                   & Rec.                                                 &                         &                         &                              &                         & \textbf{0.1}$\blacktriangle$                                                                & 0.1075                                                                      & 0.0005                                    &                                           \\
                                                                                           &                                   & F1                                                   &                         &                         &                              &                         & \textbf{0.1804}$\blacktriangle$                                                             & 0.1942                                                                      & 0.0011                                    &                                           \\ 
    \midrule
    \multirow{12}{*}{32}                                                                   & \multirow{4}{*}{I}                & Acc.                                                 & 0.5260                  & 0.5113                  & 0.5573                       & 0.5264                  & 0.5528                                                                       & 0.5362                                                                      & \textbf{0.8606}$\blacktriangle$                          & 0.5291                                    \\
                                                                                           &                                   & Prec.                                                & 0.0*                    & \textbf{1.0}$\blacktriangle$           & 0.0*                         & \textbf{1.0}$\blacktriangle$           & 0.8616                                                                       & \textbf{1.0}$\blacktriangle$                                                               & 0.0                                       & 0.0*                                      \\
                                                                                           &                                   & Rec.                                                 & 0.0                     & 0.0559                  & 0.0                          & 0.0636                  & 0.0272                                                                       & 0.0016                                                                      & 0.0                                       & 0.0                                       \\
                                                                                           &                                   & F1                                                   & 0.0*                    & 0.1058                  & 0.0*                         & 0.1196                  & 0.0528                                                                       & 0.0033                                                                      & 0.0                                       & 0.0*                                      \\ 
    \cmidrule{2-11}
                                                                                           & \multirow{4}{*}{II}               & Acc.                                                 & 0.8933                  & 0.8001                  & 0.8743                       & 0.8006                  & 0.8180                                                                       & 0.8117                                                                      & 0.8298                                    & 0.8034                                    \\
                                                                                           &                                   & Prec.                                                & \textbf{1.0}$\blacktriangleright$            & 1.0                     & \textbf{1.0}$\blacktriangleright$                 & 0.0*                    & 0.9754                                                                       & 0.9669                                                                      & 0.0                                       & 0.0*                                      \\
                                                                                           &                                   & Rec.                                                 & 0.7909                  & 0.0675                  & \textbf{0.3333}$\blacktriangleright$              & 0.0                     & \textbf{0.1175}$\blacktriangle$                                                             & 0.0563                                                                      & 0.0                                       & 0.0                                       \\
                                                                                           &                                   & F1                                                   & 0.8832                  & 0.1265                  & \textbf{0.5}$\blacktriangleright$                 & 0.0*                    & \textbf{0.2097}$\blacktriangle$                                                             & 0.1065                                                                      & 0.0                                       & 0.0*                                      \\ 
    \cmidrule{2-11}
                                                                                           & \multirow{4}{*}{III}              & Acc.                                                 &                         &                         &                              &                         & 0.7296                                                                       & 0.7206                                                                      & 0.6778                                    &                                           \\
                                                                                           &                                   & Prec.                                                &                         &                         &                              &                         & \textbf{0.8717}$\blacktriangle$                                                             & 1.0                                                                         & 0.0                                       &                                           \\
                                                                                           &                                   & Rec.                                                 &                         &                         &                              &                         & 0.0693                                                                       & 0.0804                                                                      & 0.0*                                      &                                           \\
                                                                                           &                                   & F1                                                   &                         &                         &                              &                         & 0.1285                                                                       & 0.1489                                                                      & 0.0*                                      &                                           \\ 
    \midrule
    \multirow{12}{*}{48}                                                                   & \multirow{4}{*}{I}                & Acc.                                                 & 0.5073                  & 0.5065                  & 0.5351                       & 0.5371                  & 0.5241                                                                       & 0.5035                                                                      & 0.8467                                    & 0.5010                                    \\
                                                                                           &                                   & Prec.                                                & 0.0*                    & 0.0*                    & 0.0*                         & 0.0*                    & 0.9263                                                                       & 0.0*                                                                        & 1.0                                       & 0.0*                                      \\
                                                                                           &                                   & Rec.                                                 & 0.0                     & 0.0                     & 0.0                          & 0.0                     & 0.0337                                                                       & 0.0                                                                         & 0.0001                                    & 0.0                                       \\
                                                                                           &                                   & F1                                                   & 0.0*                    & 0.0*                    & 0.0*                         & 0.0*                    & 0.0651                                                                       & 0.0*                                                                        & 0.0002                                    & 0.0*                                      \\ 
    \cmidrule{2-11}
                                                                                           & \multirow{4}{*}{II}               & Acc.                                                 & 0.7666                  & 0.6961                  & 0.6429                       & 0.7262                  & 0.7626                                                                       & 0.6797                                                                      & 0.8306                                    & 0.7004                                    \\
                                                                                           &                                   & Prec.                                                & \textbf{1.0}$\blacktriangleright$            & 0.0                     & 0.0*                         & 0.0*                    & \textbf{0.9759}$\blacktriangle$                                                                        & 0.9915                                                                      & 0.0*                                      & 0.0*                                      \\
                                                                                           &                                   & Rec.                                                 & 0.6657                  & 0.0                     & 0.0                          & 0.0                     & \textbf{0.2602}$\blacktriangle$                                                                        & 0.0965                                                                      & 0.0                                       & 0.0                                       \\
                                                                                           &                                   & F1                                                   & 0.7993                  & 0.0                     & 0.0*                         & 0.0*                    & \textbf{0.4109}$\blacktriangle$                                                                        & 0.1758                                                                      & 0.0*                                      & 0.0*                                      \\ 
    \cmidrule{2-11}
                                                                                           & \multirow{4}{*}{III}              & Acc.                                                 &                         &                         &                              &                         & 0.7258                                                                       & 0.7053                                                                      & 0.6713                                    &                                           \\
                                                                                           &                                   & Prec.                                                &                         &                         &                              &                         & 0.7954                                                                       & 1.0                                                                         & 0.0                                       &                                           \\
                                                                                           &                                   & Rec.                                                 &                         &                         &                              &                         & 0.0714                                                                       & 0.0343                                                                      & 0.0*                                      &                                           \\
                                                                                           &                                   & F1                                                   &                         &                         &                              &                         & 0.1310                                                                       & 0.0664                                                                      & 0.0*                                      &                                           \\
    \bottomrule
    \end{tabular} 
    \caption[Detection Performance with Synthetic Data and Pattern Detector]{The table shows the classification performance for \acrshortpl{ids} that participate in the proposed \gls{cids} and both utilize synthetic data and the pattern detector.}
    \label{tab:pattern_synthetic_classification}
\end{table}

\subsection{Overhead Reduction}\label{sec:overhead_reduction}
    \begin{table}[H]
        \scriptsize
        \centering
        \setlength{\extrarowheight}{0pt}
        \addtolength{\extrarowheight}{\aboverulesep}
        \addtolength{\extrarowheight}{\belowrulesep}
        \addtolength{\tabcolsep}{-0.4em}
        \setlength{\aboverulesep}{0pt}
        \setlength{\belowrulesep}{0pt}
        \setlength{\extrarowheight}{.1em}
        \begin{tabular}{cclllllll} 
    \toprule
    \multirow{2}{*}{\begin{tabular}[c]{@{}c@{}}\textbf{Hash }\\\textbf{Size}\end{tabular}} & \multirow{2}{*}{\textbf{Dataset }} & \multicolumn{2}{c}{\textbf{Simple}}                       & \multicolumn{3}{c}{\textbf{Complex}}                                    & \multicolumn{1}{c}{\multirow{2}{*}{\begin{tabular}[c]{@{}c@{}}\textbf{Sum }\\\textbf{Parameters}\end{tabular}}} & \multicolumn{1}{c}{\multirow{2}{*}{\begin{tabular}[c]{@{}c@{}}\textbf{Compression}\\\textbf{Rate}\end{tabular}}}  \\ 
    \cmidrule(lr){3-3}\cmidrule(lr){4-4}\cmidrule(lr){5-5}\cmidrule(lr){6-6}\cmidrule(lr){7-7}
                                                                                           &                                    & \multicolumn{1}{c}{Regions} & \multicolumn{1}{c}{Samples} & \multicolumn{1}{c}{Regions} & \multicolumn{1}{c}{~Samples} & Parameters & \multicolumn{1}{c}{}                                                                                            & \multicolumn{1}{c}{}                                                                                              \\ 
    \midrule
    \multirow{3}{*}{16}                                                                    & I                                  & 596                         & 50083                       & 284                         & 306167                       & 38851      & 39447                                                                                                           & 0,28\%                                                                                                            \\
                                                                                           & II                                 & 649                         & 748                         & 343                         & 77009                        & 206765     & 207414                                                                                                          & 6,84\%                                                                                                            \\
                                                                                           & III                                & 1170                        & 717                         & 341                         & 7041                         & 114569     & 115739                                                                                                          & 38,25\%                                                                                                           \\ 
    \midrule
    \multirow{3}{*}{32}                                                                    & I                                  & 3508                        & 178391                      & 231                         & 177859                       & 83390      & 86898                                                                                                           & 0,63\%                                                                                                            \\
                                                                                           & II                                 & 3911                        & 14396                       & 391                         & 63361                        & 178740     & 182651                                                                                                          & 6,02\%                                                                                                            \\
                                                                                           & III                                & 6076                        & 3646                        & 422                         & 4112                         & 178020     & 184096                                                                                                          & 60,85\%                                                                                                           \\ 
    \midrule
    \multirow{3}{*}{48}                                                                    & I                                  & 6351                        & 204021                      & 127                         & 152229                       & 66580      & 72931                                                                                                           & 0,52\%                                                                                                            \\
                                                                                           & II                                 & 7325                        & 52749                       & 262                         & 25008                        & 132562     & 139887                                                                                                          & 4,61\%                                                                                                            \\
                                                                                           & III                                & 10569                       & 4426                        & 335                         & 3332                         & 181068     & 191637                                                                                                          & 63,33\%                                                                                                           \\
    \bottomrule
    \end{tabular} 
        \caption[Parameter Number Analysis]{The table shows the compression rates for different datasets and hash sizes.}
        \label{tab:region_analysis}
    \end{table}

    In order to evaluate the reduction of the communication overhead, the number of attack samples in the local training datasets are put in relation to the number of parameters being exchanged in the \gls{cids}. Table~\ref{tab:region_analysis} shows the number of simple and complex regions that are created by adding the local training datasets to the pattern database. Additionally, the number of samples within all simple and complex regions are listed. As the only exchanged parameter in a simple region is the corresponding class label, the number of samples is equal to the number of exchanged parameters. In order to determine the number of parameters for the complex regions, a more detailed analysis of the exchanged \glspl{gmm} and projection matrices from the \gls{pca} has to be conducted. In particular, the number of components for the dimensionality reduction via \gls{pca}, the type of covariance matrix and number of components of the \gls{gmm} and the dimensionality of the input data has to be known in order to determine the number of parameters in complex regions (see Section~\ref{sec:generative_fitting}). The second to last column in the table sums the number of parameters of both simple and complex region for each combination of dataset and hash size. The last column indicates the percentage of the number of parameters exchanged in the presented approach with respect to the number of parameters in the alternative \gls{cids} (see Table~\ref{tab:num_samples_parameters}). In other words, it answers the question of how small is the amount of traffic generated by the proposed architecture compared to the alternative baseline scenario.

\end{document}