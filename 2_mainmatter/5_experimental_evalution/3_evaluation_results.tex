\documentclass[../../main.tex]{subfiles}
\begin{document}

\section{Evaluation Results}

\subsection{Baseline Experiments}


\begin{table}[H]
    \footnotesize
    \centering
    \setlength{\extrarowheight}{0pt}
    \addtolength{\extrarowheight}{\aboverulesep}
    \addtolength{\extrarowheight}{\belowrulesep}
    \addtolength{\tabcolsep}{-0.4em}
    \setlength{\aboverulesep}{0pt}
    \setlength{\belowrulesep}{0pt}
    \setlength{\extrarowheight}{.1em}
    \begin{tabular}{clllllllll} 
    \toprule
    \multirow{3}{*}{\textbf{Dataset}} & \multicolumn{1}{c}{\multirow{3}{*}{\textbf{Metric}}} & \multicolumn{8}{c}{\textbf{Removed Class}}                                                                                                                                                                                                                                                                                                                      \\ 
    \cmidrule{3-10}
                                      & \multicolumn{1}{c}{}                                 & \multicolumn{2}{c}{Brute Force}                   & \multicolumn{2}{c}{Web Attack}                         & \multicolumn{2}{c}{DoS}                                                                                                                                    & \multicolumn{1}{c}{\multirow{2}{*}{DDoS}} & \multicolumn{1}{c}{\multirow{2}{*}{Bot}}  \\ 
    \cmidrule(l){3-8}
                                      & \multicolumn{1}{c}{}                                 & \multicolumn{1}{c}{SSH} & \multicolumn{1}{c}{Web} & \multicolumn{1}{c}{SQL Inj.} & \multicolumn{1}{c}{XSS} & \multicolumn{1}{c}{\begin{tabular}[c]{@{}c@{}}HTTP\\ High Vol.\end{tabular}} & \multicolumn{1}{c}{\begin{tabular}[c]{@{}c@{}}HTTP \\Low Vol.\end{tabular}} & \multicolumn{1}{c}{}                      & \multicolumn{1}{c}{}                      \\ 
    \midrule
    \multirow{4}{*}{I}                & Acc.                                                 & 0.9783                  & 0.8769                  & 0.9087                       & 0.9643                  & 0.8659                                                                       & 0.9243                                                                      & 0.8474                                    & 0.8684                                    \\
                                      & Prec.                                                & 0.9999                  & 1.0                     & 1.0                          & 1.0                     & 0.9937                                                                       & 0.9986                                                                      & 0.0*                                      & 0.0*                                      \\
                                      & Rec.                                                 & 0.9955                  & 0.1677                  & 0.1818                       & 0.8636                  & 0.0674                                                                       & 0.5947                                                                      & 0.0                                       & 0.0                                       \\
                                      & F1                                                   & 0.9977                  & 0.2872                  & 0.3076                       & 0.9268                  & 0.1262                                                                       & 0.7454                                                                      & 0.0*                                      & 0.0*                                      \\ 
    \midrule
    \multirow{4}{*}{II}               & Acc.                                                 & 0.9551                  & 0.8445                  & 0.9206                       & 0.9191                  & 0.8906                                                                       & 0.9326                                                                      & 0.8465                                    & 0.8457                                    \\
                                      & Prec.                                                & 1.0                     & 0.0*                    & 1.0                          & 1.0                     & 0.9862                                                                       & 0.9982                                                                      & 0.0*                                      & 0.0*                                      \\
                                      & Rec.                                                 & 0.9813                  & 0.0                     & 0.3333                       & 0.6667                  & 0.0706                                                                       & 0.7825                                                                      & 0.0                                       & 0.0                                       \\
                                      & F1                                                   & 0.9905                  & 0.0*                    & 0.5                          & 0.8                     & 0.1317                                                                       & 0.8773                                                                      & 0.0*                                      & 0.0*                                      \\ 
    \midrule
    \multirow{4}{*}{III}              & Acc.                                                 &                         &                         &                              &                         & 0.7685                                                                       & 0.7316                                                                      & 0.7022                                    &                                           \\
                                      & Prec.                                                &                         &                         &                              &                         & 0.8653                                                                       & 0.9960                                                                      & 0.0*                                      &                                           \\
                                      & Recall                                               &                         &                         &                              &                         & 0.0918                                                                       & 0.1182                                                                      & 0.0                                       &                                           \\
                                      & F1                                                   &                         &                         &                              &                         & 0.1660                                                                       & 0.2114                                                                      & 0.0*                                      &                                           \\
    \bottomrule
    \end{tabular} 
    \caption[Baseline Detection Performance]{The table indicates wether a dataset includes a particular attack class (\cmark) or not (\xmark).}
    \label{tab:baseline_detection}
\end{table}

\subsection{Classification Improvement}

\subsection{Overhead Reduction}
    % after indexing
        % how many regions have been created during the indexing service?
        % how many flows are stored per region (histogram?)
        % how many complex and non-complex regions?

        % how man features after complexity reduction?

\end{document}