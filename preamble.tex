\usepackage[activate={true,nocompatibility},
final,
tracking=true,
kerning=true,
spacing=true,
factor=1100,
stretch=10,
shrink=10]{microtype}
\microtypecontext{spacing=nonfrench}

\usepackage{lmodern}
\usepackage[T1]{fontenc}
\usepackage[utf8]{inputenc}
\usepackage[english]{babel}

\usepackage[a4paper, top=2.5cm, bottom=2.5cm, inner=3.0cm, outer=2.5cm, headheight=15pt]{geometry}
\usepackage{parskip} 

\usepackage{amsmath, amssymb, amsthm, bm}
\usepackage{numprint}


\usepackage{array} % for '\newcolumntype' macro
\newcolumntype{C}{>{{}}c<{{}}} % for binary and relational operators
\newcolumntype{L}{>{\displaystyle}l} % automatic display-style math mode, left-aligned

\theoremstyle{definition}
\newtheorem{theorem}{Theorem}
\newtheorem{definition}[theorem]{Definition}

% \usepackage{titlesec}
% \titleformat{\section}
%   {\normalfont\fontsize{14}{15}\bfseries}
%   {\thesection}{1cm}{}

% \titleformat{\subsection}
%   {\normalfont\fontsize{14}{15}\bfseries}
%   {\thesubsection}{1cm}{}

\usepackage{svg}  
\usepackage{graphicx}
\usepackage{transparent}
\usepackage{eso-pic}

\usepackage[noend]{algorithmic}
\usepackage{algorithm}

\renewcommand{\algorithmicrequire}{\textbf{Input:}}
\renewcommand{\algorithmicensure}{\textbf{Output:}}
\renewcommand{\algorithmicforall}{\textbf{for each}}
\renewcommand{\algorithmiccomment}[1]{// #1}

\usepackage{caption}
\usepackage{subcaption}
\captionsetup{width=0.8\textwidth}

\usepackage[
backend=biber,
style=alphabetic,
sorting=ynt
]{biblatex}
\addbibresource{references.bib}

\usepackage{csquotes}

\usepackage[colorlinks, linkcolor=black, citecolor=black]{hyperref}
\usepackage{url}

\usepackage{fancyhdr}

\pagestyle{fancy}
\fancyhf{}
\fancyhead[LE]{\nouppercase\leftmark}
\fancyhead[RO]{Section \nouppercase\rightmark}
\fancyhead[RE,LO]{\thepage}
\fancyfoot[C]{}

\usepackage{booktabs}

\newcommand{\RomanNumeralCaps}[1]{\MakeUppercase{\romannumeral #1}}

\usepackage{tikz}
\usetikzlibrary{external}
\tikzexternalize[prefix=tikz/]
\usepackage{pgfplots}
\pgfplotsset{compat=1.17}

\newcommand{\inputtikz}[1]{%
  \tikzsetnextfilename{#1}%
  \input{#1.tex}%
}

\usepackage[toc, acronym]{glossaries}
\makeglossaries




